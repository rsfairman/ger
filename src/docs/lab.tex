
% The "lab notebook" for vcnc, the virtual G-code generator.

%\documentclass[titlepage,oneside,10pt]{article}
\documentclass{article}

% Need this for pmatrix (among other things).
\usepackage{amsmath}
\usepackage{amsfonts}

% Allows various tweaks to itemize as optional arguments.
\usepackage{enumitem}

% Needed to include graphics objects (from metapost).
%\usepackage{graphics}

%\usepackage{epic,eepic} \setlength\maxovaldiam{60pt}

% This is so that \atan2 is printed in roman type for formulas, like
% \cos, \sin, etc. Taken from the original TeX book, p. 361.
%\def\atan2{\mathop{\rm atan2}\nolimits}

\begin{document}

\raggedbottom

% These are for marginal notes on the right or left of the main text.
\newcommand{\mymargin}[1]{\marginpar{\rm\tiny #1}}
%\newcommand{\mymargin}[1]{\marginpar{\rm\scriptsize #1}}
\newcommand{\leftmar}[1]{\reversemarginpar \mymargin{#1}}

\noindent The lab notebook for \emph{vcnc}, aka \emph{Ger}. There are
three important documents: this one, {\tt manual.tex} and {\tt
  overview.tex}. The {\tt overview} document has higher level thoughts
on the problem that are too long and involved for the kind of
blow-by-blow stuff that appears here.

\section{July 22, 2022}

This is a revival of the vcnc program that I worked on years ago. John
Dunlap from the Bangor Makerspace is part of the reason I've reopened
this. Enough time has passed that certain things should be technically
easier, and my intent is to dial back on the goals too.

I spent most of today going through the old stuff. I pared and
combined the various old versions without getting too obsessed with
eliminating every last redundancy. That stuff is in the {\tt ancient}
directory. I pulled the documentation out of there, and
renamed many of these files, to be available here (outside of {\tt
  ancient}) for reference since certain aspects of the documentation
remain relevant. There were essentially three versions.

\begin{itemize}[itemsep = -3pt,leftmargin=*]
\item \emph{G-code Interpreter} was (roughly) the first attempt. See {\tt
  labA.tex}. I started this in November, 2008 and the lab notebook
  runs through the end of June, 2009. It looks like I let Charlie
  Whorton play with it and had a meeting with Brian Barker. That
  meeting went nowhere and I think that was part of the reason the
  idea fizzled out.
\item \emph{revived} appears to have been a brief attempt at building on
  the previous version. It looks like I made a fair number of fairly
  extensive changes, but it doesn't look like I left a lot of notes
  about it.
\item \emph{qt dev} was the last significant thing I did. I think (?)
  that I started on revived, in Java, then decided to port it
  all to C++ and Qt. I think I was hoping that C++ would be faster and
  allow me to access the GPU more easily. It looks like I worked on
  this from April to June of 2013. Based on date stamps, I think (?) I
  may have worked on revived a bit in 2011, then decided to go the
  C++ route in 2013.

  Based on comments found in this version, it looks like I did fix
  some non-trival bugs that were in the Java -- nothing
  earth-shattering, but they were important improvments.
\item \emph{johns CNC} was a quick revision of revived so that John
  Dunlap could look at it. This can be ignored.
\end{itemize}

For all the shortcomings of Java, and even after all these years, I
still don't see a better choice than Java/Swing. That's kind of
shocking, but seems to be true. One reason not to use fully compiled
language, like C++, is that I want to allow the user to define new CNC
commands. Doing that with C++ would be very difficult for anyone who's
not completely familiar with C++ and all it entails. JavaScript might
work too, but complex data structures are difficult to work with in
JS. OTOH, it may be that there are already tools to do certain
rendering tasks in that framework.

Anyway, the goals for this attempt are as follows
\begin{itemize}[itemsep = -3pt,leftmargin=*]
\item Allow the user to input standard G-code, pretty much as in every
  previous version.
\item Display some kind of 3D (or 3D-ish) rendering of the result. I
  hope that the tech (outside libraries) has reached a point where I
  can do this with a reasonable amount of effort.
\item Allow the user to define ``wizards'' to extend standard
  G-code. I've considered the idea of extending G-code to include
  various programming abilities, sort of like Macro-B, but it seems
  like a sink-hole. If a user knows enough to ``program,'' then he
  knows enough to figure out how to do it via Java. This leaves the
  G-code as something relatively clean.
\item Output ``minimal'' G-code, meaning G-code that only uses the
  most basic commands, like {\tt G00}, {\tt G01} and so forth. In
  earlier versions, a goal that I had in my mind was sending pulses to
  steppers, but that problem has been (more or less) solved by things
  like GRBL.
\item Don't worry about multi-axis machines. The average hobbyist
  doesn't have that. At the same time, tools that undercut raise many
  of the same issues and I do want to consider such tools.
\item It should be possible to simulate (\emph{i.e.}, watch) the
  motion of the tool as it cuts. Ideally, there should be a way to
  indicate where tool crashes might occur against fixtures or if an
  attempt is made to run the tool through the material with rapid
  travel.
\item John Dunlap wants to be able to output STL files. Since these
  are essentially just a list of the triangles defining the surface of
  the object, doing this should be a small extension to the ability to
  redner in 3D -- assuming that kind of rendering is possible.
\end{itemize}
Aside from improving the rendering and adding the use of wizards, this
is pretty much what already exists.

Here's a summary of important points from {\tt labA}.
\begin{itemize}[itemsep = -3pt,leftmargin=*]
\item From the very beginning, I was thinking of rendering with a QTD
  (quad-tree with depth). In fact, I started with what I called a DA
  (depth array), which I think was just a 2D array of shorts --
  hoggish of memory, but fast.
\item I messed around implementing a BSP (binary search partition) and
  Gouraud shading, but I think it was too slow or something. Another
  idea was a kind of z-buffer with check for hidden pixels.
\item It looks like I must have been working with some
  kind of simple G-code interpreter while I was working out some of
  the kinks with rendering; then I improved the renderer. In fact,
  long after I had (or thought I had) more or less settled various
  rendering issues, I was implementing things like cutter comp.
\item There's some useful algrebra in {\tt labA} about determining the
  intersection of simple curves.
\item I had various difficulties dealing with cutter comp, and I need
  to make sure that what I did is correct.
\end{itemize}
From {\tt labB}, I see the following.
\begin{itemize}[itemsep = -3pt,leftmargin=*]
\item I discarded any attempt to deal with undercutting.
\item I found some way (?) to deal with cutter comp that is
  analytical. See the entry for May 13, 2013.
\end{itemize}

My goal should be to produce a triangulation, and do it in a way that
allows for undercutting. This requires (for practical reasons) that I
use some outside library to render the triangles. I do still want a
version that uses something like a DA, both for testing and because it
will often be exactly what the user needs. If you're cutting shapes
out of plywood, then you don't need a proper 3D rendering.

Note that OpenSCAM is now CAMotics.

\section{July 23, 2022}

Continue reading old documentation, and see how the software landscape
has changed.

I mentioned above that CAMotics is the new OpenSCAM. The author of
CAMotics posted an interesting overview of his journey:

\noindent\verb=medium.com/buildbotics-blog/is-camotics-alive-3be83275493=

\noindent Most of the above is a kind of teaser for a project he is
being purposely vague about, but that he says will allow him to make a
bit of money off the whole thing, while keeping it open-source. The
above was written in 2016, and I don't see anything on the current
CAMotics website that could be this project. The
interesting part of the above is that people do donate on his page,
but not very much; he says that it works out to about \$1.33 per
hour of his time. Another interesting thing is that he released this
in 2012, and started working on it in 2011. CAMotics is available on
github. See

\noindent\verb=github.com/CauldronDevelopmentLLC/CAMotics=

\noindent or his original version (called OpenSCAM, haha) is at

\noindent\verb=github.com/jwatte/OpenSCAM=.

Somebody put together a list of open-source CNC software:

\noindent\verb=www.reddit.com/r/CNC/comments/aizatc/free_and_open_source_camcnc_software/=

\noindent although most of what's on there isn't so
interesting. F-Engrave is one exception. It's limited, but I've used
it and it does it's thing well.

Something I found seperately is Open Cascade Technology (OCCT), which
I think was/is used by HeeksCAD. I see from the Wikipedia page, that
several well-known programs use this, like FreeCAD and KiCad. There
was a fork of this, back in 2011, called Open Cascade Community
Edition. The community edition is at \verb=github.com/tpaviot/oce= and
the non-community edition is at
\verb=github.com/gullibility/OpenCASCADE=, although I'm not sure how
``official'' that is. These are in C++, although there seems to be
some kind of Java wrapper. This might be a way to go, but it's awfully
heavy.

Better would be something designed from the ground up with Java in
mind, like Java OpenGL (JOGL). There's something called Java3D, but I
think that's more about rendering entire scenes and is higher-level in
some sense and is more game-oriented. Although it still exists, I
think it is dead, practically speaking.

Another choice is libGDX, which John Dunlap was advocating. This might
work, but it's a huge thing intended for games. That said, it might be
possible to use only the lower-level stuff that is needed. If you look
on their website, under ``features,'' their ``low-level OpenGL
helpers'' might be all I need.

It seems that libGDX is based on LWJGL (Light-Weight Java Game
Library). Wikipedia says that LWJGL is the ``backend'' used by libGDX
for ``the desktop'' (as opposed to web apps).

Ten years ago, there was something
called ``Project Sumatra'' that was supposed to get the JVM running on
GPU, but it seems to have died.

\verb=blogs.oracle.com/javamagazine/post/programming-the-gpu-in-java=
isn't all that informative, but worth a quick look.

Another interesting thing is JCSG (Java Constructive Solid Geometry),
at \verb=github.com/miho/JCSG=. It might (?) be possible to use this
to go from individual tool paths (like from a single line of G-code)
to a triangulation. It claims to work using BSPs.

LWJGL is probably the place to start; either that or JOGL, but LWJGL
seems to be more active, particularly since libGDX is based on it and
libGDX is extremely active. And consider JCSG.

At the momemnt, I am thinking that the following might be a good way
to approach the problem of triangulation and sweeping out the tool
path. The basic idea is to use an octree, but, instead of going down
to the level of tiny $0.001\times 0.001$ voxels, if a cube of the
octree is partially filled, let the cut through the cube be
represented by one or more triangles. Of course, you could let the
entire thing be represented by a single partially filled cube with a
huge number of triangles -- that is sort of the goal after all. The
point of this idea is that the set of triangles in a give cube would
divide the cube into what's inside and what's outside the solid. By
using an octree this way, the sweeping should be faster. Or maybe
JCSG could do this, and I won't have to think about it. If JCSG uses
BSPs, and does it well, then I don't think I can really do things any
faster. 

I downloaded the JCSG stuff from github, just to remind myself that
it's on the table. LWJGL (version 3) too. ligGDX is more of a monster,
so I didn't download that.

I'm starting on {\tt v01}.

How to set up the UI? Clearly, I want tabs so that the user can open
multiple G-code files, but what about the output, bearing in mind that
there could be a variety of outputs? Before, I think that any output
opened to a different window, with one window for each instance of
``output.'' There could be a variety of outputs: the transpiled
G-code, which could be transpiled to varying degrees; different kinds
of rendering (2.5D, 3D, etc.). The user might want to look at two or
more of these at the same time, and that leads me to want to put
things in individual windows. OTOH, having a zillion windows open is
annoying. One way might be to put everything into its own tab, but
allow the user to ``windowize'' any individual tab. This gives the
user control over how messy things get, while allowing him to compare
things as he sees fit.

I think I could do this using the same basic code, whether the thing
being displayed is in a tab or a window. I also like this because it
keeps things neat. So I need a tabbed area that can display arbitrary
things in the tabs. I don't think that's hard, since additions to a
{\tt JTabbedPane} consists of {\tt Component} objects.

G-code needs to be contained in a particular way. Every G-code file
can have various settings associated with it, like a tool table,
choice for metric/inch, and the like. Given how this will be used most
of the time, each G-code file needs to point to the window that
contains it because that window has the settings for these items. It's
tempting to put these settings directly in the G-code somehow, within
specially formatted comments, but some of these items would be
difficult to express in ordinary text done in such a way that it would
make sense to the average user -- and so that it wouldn't be a big
pain to maintain. What could be done is have the main program save a
set of various named settings as ``persistent data,'' and let the text
of the G-code refer to these settings.

So, if a particular G-code file is opened in a given window, and that
G-code does not refer to a particular set of settings, then it will
use the settings for that window. This requires that it be possible to
set all these values ``together'' so that each collection of settings
can be named for future reference.

\section{July 24, 2022}

Have the rough plan in place for the UI. Start bringing over the
transpiler. The Lexer part came over with almost no changes. I did
make use of some improvements to how comments are read in the C++ code
versus how it was done in Java in revived.

Something I debated was how to handle the situation where the user has
a given bit of G-code and creates various outputs from it, like
transpiled code or a rendering. It is tempting to let them keep old
versions in tabs so that they can compare changes, but that starts to
get confusing (for me to program and for the user). Instead, only
allow one ``kind'' of output from every input. If they have (say)
lexer output from their g-code and they regenerate the lexer output,
then replace the contents of the existing lexer window.

I would like the ability to drag tabs around to reorder them, but it's
painful. 

\section{July 25, 2022}

Working on how to make tabs draggable, etc. If I put this off until I
have more of the G-code infrastructure done, I will probably never do
it. I found a couple of examples -- MIT License, so free to use,
though by the time I'm done, there will be nothing original left.

After some thought, I am not going to use scrollable tabs. If the user
has lots of tabs, then it will be easier for him to see them if
they're all visible, even if they end up being ``stacked.'' There are
pros and cons, but this will be better for the user and better for me,
as a programmer. The only downside I see is that you lose some
vertical drawing area if you have a lot of tabs stacking up.

I spent a fair amount of time unravelling the drag \& drop code I
found on github for tabbed panes. I could do a lot more to clean it
up.

\section{July 27, 2022}

I've made some progress on the problem of getting the tabbed area to
work the way I want, although I haven't settled on exactly what it is
that I want. Something I've given up on is detecting double-clicks in
the tabs as a way to bring tabs to their own window. No doubt, there's
a way to do it, but as things stand, the drag \& drop stuff interferes
with detecting clicks.

What I'm leaning toward now is allowing multiple windows, each with
all the menus and so forth. In practice, this would be treating the
various tabs as though they all belong to the same window. I would
like to be able to DnD tabs from one window to another, but that may
be tricky.

Can I do DnD between windows? How about allowing you to make a tab ``dead?''

\section{August 8, 2022}

Had a bit of a hiatus, and spent a few days without updating this
document. I've moved to v02 because the drag \& drop stuff is working
well enough for the time being, and might be fully done. I deleted a
bunch of versions of the intermediate D\&D code going forward, but
those versions are saved with v01.

Next: set up the lexer to accept externally defined Java
functions. The big issue is how to distinguish ordinary G-code from these new
functions. There are several ways this could be done, but I will say
that these new functions have to start with a lower-case letter. Then,
open parenthesis, comma-delimited arguments, close
parenthesis. Another way would be to say that these must all start
with something like {\tt @}, but that's painful for the user to type
all the time. 

\section{August 9, 2022}

After some discussion with John Dunlap, I've revised my opinion about
external functions. Allow names for an external to start with any
letter. All I need to do is peek ahead one letter to see whether we
have a single-letter G-code or a wizard function.

One problem is that everything in strict G-code is of the form
letter+something. For example, you can't have a bare number, like {\tt
  2.0} without a letter in front of it. This makes arguments to wizard
functions trickier, and it also means that the tokenizer can no longer
be context-free (or not quite as context-free). 

\section{August 11, 2022}

I've settled on making a few significant changes. One of them is
allowing codes (G or M) that are given using floating-point
numbers. Most codes, use a whole number, like {\tt G01}, but there are
some oddballs like {\tt G84.2} for tap drilling. I am trying to be
open-ended about what the system will accept, so I need to accept
these, even if I don't do anything with them. It's up to the user to
define a wizard to interpret it.

\section{November 2, 2022}

I got sucked into getting proper bylaws for the makerspace, and then
being president. Hopefully that stuff is on a more even keel and I can
get back to this.

One of the things I see now is that things like tool crashes and the
geometry of the (simulated) physical machine are too fiddly to deal
with. The relates to things like {\tt G28} (machine zero return) and
the like. 

\section{November 6, 2022}

I got the line numbers to display, but there are (minor) bugs. And I
got the first (zero-th) layer in place to eliminate subroutine
calls. Next is to get some of the preperatory commands in place -- for
things like setting up the tool table and the like.

The commands I need to handle before the G-code proper are:
\begin{itemize}
\item Something for tool table, although this is tricky for unusual
  tools. I need this for cutter comp at a minimum, but also for the
  shape of the tools.
\item Work offsets table.
\item Billet dimensions, with margin
\end{itemize}
Some of this data (work offsets and tool table) should be persistent
between runs of the program. There is a {\tt Preferences} class to
help with this, but it is really intended for much shorter
``settings,'' not entire data structures. This could be used here, but
it probably makes more sense to let the user save a settings file (or
more than one) with the settings he wants. On one hand, it will
usually be the case that he only wants one such settings file; but he
might want different files for different cases. The best solution
might be for there to be a default location for the settings file, but
allow the user to choose another file.

\section{November 21, 2022}

All the layers are in place, including the basic wizard
framework. Most of the machine setup stuff hasn't been done (e.e., tool
table) and none of the rendering.

Now that the bulk of the code is there, I've started going back and
cleaning things up. One UI thing is toggling line numbers. I have
something that sort of works, but it's buggy. What I want is something
that acts almost identically to {\tt JTextArea}, but with the ability
to toggle line numbers.

\section{August 8, 2022}
\section{August 8, 2022}
\section{August 8, 2022}
\section{August 8, 2022}
\section{August 8, 2022}
\section{August 8, 2022}
\section{August 8, 2022}




\end{document}
