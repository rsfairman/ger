
\documentclass[titlepage,oneside,10pt]{article}

% So that I don't have to type this all the time.
\newcommand{\ger}{\textsc{Ger}}

% Need this for pmatrix (among other things).
\usepackage{amsmath}

% Needed to include graphics objects (from metapost).
\usepackage{graphics}

\usepackage{epic,eepic} \setlength\maxovaldiam{60pt}

% For ``long'' tables.
\usepackage{longtable}

% To make a short hrule as a divider between things that aren't quite
% worthy of being sections. The \vskip 0mm is to put tex into vertical
% mode. For some reason, it seems to put a little extra space above
% the hrule -- probably because it's relative to the baseline of the text.
%\newlength{\rulewidth}
%\newcommand\mtrule{\hrule width \rulewidth}
%\newcommand\setrulewidth[1]{\setlength{\rulewidth}{#1}}
\newcommand\rulediv{\vskip 0mm\hfil\rule{0.4\textwidth}{0.4pt}\hfil\vskip 1mm}


\begin{document}

\raggedbottom

% These are for marginal notes on the right or left of the main text.
\newcommand{\mymargin}[1]{\marginpar{\rm\tiny #1}}
%\newcommand{\mymargin}[1]{\marginpar{\rm\scriptsize #1}}
\newcommand{\leftmar}[1]{\reversemarginpar \mymargin{#1}}

This is a manual for developers.
There are two aspects to the code: the user-interface and the G-code
translator. The UI is basically how Swing is used, plus the plumbing
that holds the entire program together, while the translator
is the central task the program is intended to accomplish.

\section{User-interface and Plumbing}

The entry point for the program is found in\footnote{``vcnc'' stands
for ``Virtual CNC.''}
\begin{verbatim}
vcnc.Main
\end{verbatim}
There's not a lot to say about this class: it merely kicks off the entire
program by creating a 
\begin{verbatim}
vcnc.MainWindow
\end{verbatim}
This is where all the plumbing comes together and is the primary
window for user activity. Most of what happens in the class is
standard stuff: menus, windows, \emph{etc}.

Alternatively, the {\tt main()} method can invoke a series of units
tests. Which of the two is done (ordinary GUI or unit test) depends on
which of two methods is called by {\tt main()}.\footnote{It wouldn't
be hard to the choice of which to run into a command-line argument,
but it seems like an unnecessary complication.} The unit tests are
discussed further, below, with the translator.

This window puts each file in a tabbed frame. By default, Java's {\tt
  JTabbedPane} doesn't have all the functionality desired, so there's
a fair amount of UI code in the {\tt vcnc.ui.TabbedPaneDnD} package, which is
discussed further, below.

\rulediv

In addition, each tab shown has a type that corresponds to the
contents of that tab. These types are managed by the {\tt
  vcnc.ui.TabMgmt} package. What is does is simple.

\begin{verbatim}
vncc.ui.TabMgmt.TabbedType
\end{verbatim}
is an {\tt enum} listing the possible types of tab contents and
\begin{verbatim}
vncc.ui.TabMgmt.TypedDisplayItem
\end{verbatim}
is an interface that each tab {\tt Component} must implement to
associate a type with its contents. Currently, all of these extend
{\tt JScrollPane}, but any number of other {\tt awt.Component}
objects would probably work as the base class.

Currently, there are only three of these types:
\begin{verbatim}
vncc.ui.TabMgmt.GInputTab
vncc.ui.TabMgmt.LexerTab
vncc.ui.TabMgmt.ParserTab
\end{verbatim}
The {\tt LexerTab} and {\tt ParserTab} are essentially identical,
\footnote{Clean that up.} and are used to display G-code output, which
is not editable. The {\tt GInputTab} is a bit more complicated because
this code is editable and due to toggling line numbers.

The final class found in this package is 
\begin{verbatim}
vncc.ui.TabMgmt.StaticWindow
\end{verbatim}
which is used to convert a tab to an independant window whose contents
can no longer be edited.\footnote{This is also similar to {\tt
  LexerTab} and {\tt ParserTab}.}

\rulediv

The remaining files in the {\tt vcnc} package are 

\begin{verbatim}
vcnc.TextInputDialog
vcnc.TableMenu
\end{verbatim}
The {\tt TextInputDialog} class is currently used to set the material
billet, but it's not being done in the way that it ultimately should
be done.

{\tt vcnc.TableMain} is leftover junk -- test code for {\tt
  JTables}. Get rid of it.

\rulediv

TALK ABOUT HOW THE MAIN WINDOW INVOKES THE UNDERLYING TRANSLATOR.

\subsection{{\tt vcnc.ui.TabbedPaneDnD} Package}

Two features beyond those provided by {\tt JTabbedPane}
are added: an ``X'' to close the tabs, and the ability to drag and
drop the tabs within each window or from one window to another. The
close-box is managed (mostly) through {\tt ButtonTabComponent}, and
the remaining classes manage drag \& drop, with {\tt TabbedpaneDnD}
being the main class of interest externally. See\footnote{I wrote this
ages ago, and it seems to work, but it's not pretty. I need to rewrite
it, make it more of a stand-alone thing that is useful more generally
and provide better documentation for it. It's messy Swing, so I'm
putting it off. }
\begin{verbatim}
vcnc.ui.TabbedPaneDnD.ButtonTabComponent
vcnc.ui.TabbedPaneDnD.GhostGlassPane
vcnc.ui.TabbedPaneDnD.TabbedPaneDnD
vcnc.ui.TabbedPaneDnD.TabDragGestureListener
vcnc.ui.TabbedPaneDnD.TabDragSourceListener
vcnc.ui.TabbedPaneDnD.TabDropTargetListener
vcnc.ui.TabbedPaneDnD.TabTransferable
vcnc.ui.TabbedPaneDnD.TabTransferPacket
\end{verbatim}

\subsection{Miscellany in the {\tt vcnc.util} Package}

\begin{verbatim}
vcnc.util.ChoiceDialogRadio
vcnc.util.ClickListener
vcnc.util.EmptyReadException
vcnc.util.FileIOUtil
vcnc.util.LoadOrSaveDialog
vcnc.util.StringUtil
\end{verbatim}

Not much to say about these...{\tt ClickListener} may be trash, and
they could all be tidied up with unused stuff taken out. They're
mostly older code pulled from other projects.

\begin{verbatim}
\end{verbatim}
\begin{verbatim}
\end{verbatim}
\begin{verbatim}
\end{verbatim}

\section{G-code Translator}

\ger\ is similar to a (very simple) compiler or interpreter. It
converts an input file of G-code to a simpler from. This
simplification happens as the code passes through a series of layers,
where each layer handles a particular aspect of the
simplification. The code related to this translation process is all in
{\tt vcnc.tpile.*}, either in that package or in a sub-package.

The lowest layer is the lexical analyizer (or ``lexer''). Because
G-code is so simple, with very little context-dependence, the lexer is
equally simple. It converts the incoming text file to a stream of {\tt
  Token} objects. Each token represents one of the letter codes (G, M,
I, J, \emph{etc}.) and any associated value. \ger\ allows the user to
extend ordinary G-code with user-defined functions, and these
functions are also converted to {\tt Token} objects.

The {\tt Token} objects are passed to the next layer, which is the
parser. The parser assembles the tokens into {\tt Statement}
objects. These statements correspond to the individual conceptual
steps of the program. The remaining layers convert these statements
into an increasingly stripped-down subset of the possible
G-codes, ultimately producing nothing but {\tt G00, G01, G02, G03}
codes for moving the cutter, plus a few M-codes and other codes that
pass through the process untouched.

The remaining layers are where the meat of the simplification
occurs. By layering the simplification process in this way, each layer
is made easier to understand, although it does lead to a certain
amount of repetative boilerplate. These layers are given numbers,
starting with {\tt 00}.

\subsection{\tt Lexer}

The code used for lexing is found in {\tt vcnc.tpile.lex}. Classes outside
this package -- the next layer of the translator -- need access to the
{\tt Lexer} and {\tt Token} classes. The other classes here have
default visibility, so are not accessible outside the package: there's
a token buffer and some internally used exception classes.

\subsection{\tt Parser}

The parser takes a stream of {\tt Token} objects, generated by {\tt
  Lexer}, and produces a series of {\tt Statement} objects. This code
is found in {\tt vcnc.tpile.parse}. Most of the classes in this
package extend {\tt StatmentData}. Each type of {\tt Statement} may
need different data, one class for ciruclar interpolation, one for
linear moves, \emph{etc}. The names for these classes start with
``{\tt Data}.''


\subsection{\tt Layer00}

This layer eliminates all calls to subroutines. In particular, {\tt
  M98} (call subprogram) and {\tt M99} (return from subprogram) are
replaced by the code of the corresponding subprogram(s).

IN ADDITION, this also eliminates certain items that serve no purpose
in a simulator, like {\tt M07, M09} and {\tt M09} for coolant
control, together with {\tt M40} and {\tt M41} for spindle high/low
and {\tt M48} and {\tt M49} for feed and speed overrides.

For several other commands, it's not clear what the appropriate action
should be, so they are treated as a ``halt.'' These codes are {\tt
  M00, M01} and {\tt M02} (various forms of ``stop''), along with {\tt
  M47} (repeat program).

BUG: I suspect that some of these should pass all the way through the
program, and only be dropped at the very end, when rendering.

\subsection{Unit Tests}

As noted above, the unit tests can be run from the {\tt main()}
method. The basic idea is that each layer of the translator, starting
with the lexer, is able to produce output as text (a {\tt
  String}). Each test is specified by an input file, which is run
through the translator up to a certain level. The output from this run
is compared to a static text file to see if they match.

\section{Geometry}

There are a few geometric problems that come up. 

One problem is that we have two end-points of an arc and the radius,
and we need to know the center. This arises in the {\tt ArcCurve}
class. Work in the $xy$-plane. Let the two
end-points be $(x_0,y_0)$ and $(x_1,y_1)$, with radius $r$, and the
center be at $(c_x,c_y)$. We want to determine the center given the
other values.

We have two equations and two unknowns:
\begin{eqnarray*}
  r^2 &=& (x_0-c_x)^2 + (y_0-c_y)^2 \\
  r^2 &=& (x_1-c_x)^2 + (y_1-c_y)^2 \\
\end{eqnarray*}
Solve the first equation for $c_x$:
\begin{eqnarray*}
  (x_0-c_x)^2 &=& (y_0-c_y)^2 - r^2\\
  x_0-c_x &=& \pm\sqrt{(y_0-c_y)^2 - r^2}\\
  c_x &=& x_0 \mp\sqrt{(y_0-c_y)^2 - r^2}\\
\end{eqnarray*}
Substitute that into the second equation and solve for $c_y$:
\begin{eqnarray*}
  r^2 &=& (x_1-c_x)^2 + (y_1-c_y)^2 \\
  r^2 &=& \left(x_1-\left(x_0 \mp\sqrt{(y_0-c_y)^2 - r^2}\right)\right)^2 + (y_1-c_y)^2 \\
  r^2 &=& \left(x_1-x_0 \pm\sqrt{(y_0-c_y)^2 - r^2}\right)^2 + (y_1-c_y)^2 \\
  r^2 &=& (x_1-x_0)^2 \pm 2(x_1-x_0)\sqrt{(y_0-c_y)^2 - r^2} +
  (y_0-c_y)^2 - r^2 + (y_1-c_y)^2 \\
\end{eqnarray*}

Solving that for $c_y$ might work, but it's sort of
horrifying. Instead, use the fact that a line perpendicular to a chord
of the arc must pass through the center. That is, there is a chord
passing through $(x_0,y_0)$ and $(x_1,y_1)$. Let $(m_x,m_y)$ be the
mid-point of that chord. The slope of the chord is given by
$$s = {y_1-y_0\over x_1-x_0}$$
(and the order of the $x_i$ and $y_i$ doesn't matter). The slope of
the perpendicular to the chord is $-1/s$ or
$$s = {x_0-x_1\over y_1-y_0}$$
and the line perpendicular to the chord is given by
$$y - y_0 = \left({x_0-x_1\over y_1-y_0}\right) (x-x_0).$$
In particular, the center of the circle lies on this line, so we must
have
$$c_y - y_0 = \left({x_0-x_1\over y_1-y_0}\right) (c_x-x_0).$$

Let's simplify this mess...Define
\begin{eqnarray*}
  m_x &=& x_0-x_1\\
  m_y &=& y_0-y_1.
\end{eqnarray*}
The slope of the chord is then $s = m_y/m_x$ and the slope of the
perpendicular is $-1/s$ or $-m_x/m_y$. The equation of the perpendicular is
$$y - y_0 = (-m_x/m_y) (x-x_0),$$
and we must have
$$c_y - y_0 = (-m_x/m_y) (c_x-x_0),$$
or
$$c_y  = y_0 - (m_x/m_y) (c_x-x_0).$$
Substitute that into our first equation and get
\begin{eqnarray*}
  r^2 &=& (x_0-c_x)^2 + (y_0-c_y)^2 \\
  r^2 &=& (x_0-c_x)^2 + \left(y_0-\left[y_0 - (m_x/m_y) (c_x-x_0)\right]\right)^2 \\
  r^2 &=& (x_0-c_x)^2 + \left((m_x/m_y) (c_x-x_0)\right)^2 \\
  r^2 &=& (x_0-c_x)^2\left(1 + (m_x/m_y)^2\right) \\
  r^2 &=& (x_0-c_x)^2\left({m_x^2+m_y^2\over m_y^2}\right) \\
\end{eqnarray*}
so that
\begin{eqnarray*}
  (x_0-c_x)^2 &=& r^2 \left({m_y^2\over m_x^2+m_y^2}\right) \\
  x_0-c_x &=& \pm r \left({m_y\over \sqrt{m_x^2+m_y^2}}\right) \\
\end{eqnarray*}
or
\begin{eqnarray*}
  c_x &=& x_0 \mp r \left({m_y\over \sqrt{m_x^2+m_y^2}}\right) \\
\end{eqnarray*}
and plug that into the equation for $c_y$ too:
\begin{eqnarray*}
  c_y &=& y_0 - (m_x/m_y) (c_x-x_0) \\
  &=& y_0 - {m_x\over m_y} \left(x_0 \mp r \left({m_y\over
    \sqrt{m_x^2+m_y^2}}\right) - x_0\right) \\
  &=& y_0 \pm {m_x\over m_y} \left({m_yr \over \sqrt{m_x^2+m_y^2}}\right) \\
  &=& y_0 \pm {m_xr \over \sqrt{m_x^2+m_y^2}} \\
\end{eqnarray*}



\end{document}
