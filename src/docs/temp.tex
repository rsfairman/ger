
% Discussion of different way to go about writing a G-code simulator.

\documentclass[titlepage,oneside,10pt]{article}

% Need this for pmatrix (among other things).
\usepackage{amsmath}

% Needed to include graphics objects (from metapost).
\usepackage{graphics}

\usepackage{epic,eepic} \setlength\maxovaldiam{60pt}

% So that I don't have to type this all the time.
\newcommand{\ger}{\textsc{Ger}}


% For ``long'' tables.
\usepackage{longtable}

\begin{document}

\raggedbottom

% These are for marginal notes on the right or left of the main text.
\newcommand{\mymargin}[1]{\marginpar{\rm\tiny #1}}
%\newcommand{\mymargin}[1]{\marginpar{\rm\scriptsize #1}}
\newcommand{\leftmar}[1]{\reversemarginpar \mymargin{#1}}


\section{Coordinate Systems}

The coordinate system is inherently tied up with the particular
physical machine on which the G-code is intended to run. Codes like
{\tt G28} (machine zero return) are impossible to interpret in a way
that would be consistent with every machine. Various codes related to
the coordinate system are accepted by \ger, but ``translated away.''

Machines typically have several notions of coordinate system
origin. In theory, ``machine zero'' (or ``home'') is a 
position fixed at the factory\footnote{See Peter
Smid, \emph{CNC Progrmming Handbook}, 2008 (3rd ed.), p. 153, where he
says that machine zero
\begin{quote}
  ...is the position of all machine slides at one of the extreme
  travel limits of each axis. Its exact position is determined by the
  machine manufacturer and is not normally changed during the machine
  working life.
  \end{quote}
In my experience that is not always true. In any case, it would be
impossible for \ger\ to reflect the quirks of every possible machine. }
and {\tt G28} should return the cutter to that position. Another form
of origin is the Part (or Program) Reference Zero (PRZ). The PRZ is
the coordinate system relative to which most commands are given.

\ger\ defines machine zero and the (initial) PRZ to be the location of
the tool immediately before the first line of the program. As with
most real machines, the location of machine zero can't be changed. So,
the tool always starts at (X,Y,Z) = (0,0,0). \ger\ \emph{does not
accept the} {\tt G28} {\tt command}; there's very little reason to
return the tool to (0,0,0), and it's always possible to move there
with {\tt G00} or {\tt G01} in the usual way.

The important point is that \ger\ will produce code with all coordinates
given (in absolute terms) using the location of the tool at the start
of the first line of the program as the orgin. To run the output code
on a physical machine, the easiest thing do will {\it usually} be to
set the PRZ to the surface of the lower-left corner of the part. As
always, consider the possibility of interference and tool crashes.

\subsection{Changing the PRZ}

Use {\tt G52} to make the given values, expressed relative to machine
zero, the new PRZ. Thus, 
\begin{verbatim}
G52 X2.000 Y-4.000 Z1.00
\end{verbatim}
means that the location 2 units to the right, 4 units below and 1 unit
higher than machine zero will be treated as the PRZ in the code that
follows. Saying
\begin{verbatim}
G52 X0 Y0 Z0
\end{verbatim}
(or simply {\tt G52}, with no arguments) resets the coordinate system
to what it was when the program started.

The {\tt G54-G59} commands work the same way, except that the
arguments come from the work offsets table. That is, {\tt G54} (no
arguments) is the same as
\begin{verbatim}
G52 Xx Yy Zz
\end{verbatim}
where {\tt x, y} and {\tt z} are taken from the work offsets table. 

Use {\tt G92} to indicate that the given values should be used as the
current coordinates under a new PRZ. For example, 
\begin{verbatim}
G92 X1.000 Y-1.000 Z2.000
\end{verbatim}
means to reset the PRZ so that the cutter's current location is given
by $(1,-1,2)$. Saying 
\begin{verbatim}
G92 X0 Y0 Z0
\end{verbatim}
(or simply {\tt G92}, with no arguments) is a convenient way to make
the current tool location the new PRZ.

\subsection{Tool Length Offset}

The {\tt G40}, {\tt G43} and {\tt G49} commands are used on many
machines for tool length offset (TLO). This is used to adjust for the
fact that, when tools are stored in a turret, the location of the
cutting edges of the various tools will differ. TLO can also be used
to take cutter wear into account. 

\ger\ accepts these commands, but they pass through unchanged, without
affecting the surrounding code. In theory, \ger\ could have been
written to simulate the effect of the TLO commands in more detail, but
doing so would require that the tool table reflect the irritating
real-world quirks of various cutters and turrets. The value of \ger's
translation and simulation is that it allows you to avoid these
irritations -- or at least, to avert your eyes until they can't be avoided. 











\end{document}





\end{document}
