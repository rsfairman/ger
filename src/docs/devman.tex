
\documentclass[titlepage,oneside,10pt]{article}

% So that I don't have to type this all the time.
\newcommand{\ger}{\textsc{Ger}}

% Need this for pmatrix (among other things).
\usepackage{amsmath}

% To use \mathbb for blackboard bold fonts.
\usepackage{amssymb}

% Needed to include graphics objects (from metapost).
\usepackage{graphics}

\usepackage{epic,eepic} \setlength\maxovaldiam{60pt}

% For ``long'' tables.
\usepackage{longtable}

% To make a short hrule as a divider between things that aren't quite
% worthy of being sections. The \vskip 0mm is to put tex into vertical
% mode. For some reason, it seems to put a little extra space above
% the hrule -- probably because it's relative to the baseline of the text.
%\newlength{\rulewidth}
%\newcommand\mtrule{\hrule width \rulewidth}
%\newcommand\setrulewidth[1]{\setlength{\rulewidth}{#1}}
\newcommand\rulediv{\vskip 0mm\hfil\rule{0.4\textwidth}{0.4pt}\hfil\vskip 1mm}


\begin{document}

\raggedbottom

% These are for marginal notes on the right or left of the main text.
\newcommand{\mymargin}[1]{\marginpar{\rm\tiny #1}}
%\newcommand{\mymargin}[1]{\marginpar{\rm\scriptsize #1}}
\newcommand{\leftmar}[1]{\reversemarginpar \mymargin{#1}}

NOTE: I CHANGED ALL THE PACKAGES TO BE CALLED {\tt ger} INSTEAD OF {\tt vcnc}.

This document is for people who want to understand how the program
works, with an eye to contributing to it. The math part isn't too bad,
but the other parts need work, and are out of date in any case.


This is a manual for developers.
There are two aspects to the code: the user-interface and the G-code
translator. The UI is basically how Swing is used, plus the plumbing
that holds the entire program together, while the translator
is the central task the program is intended to accomplish.

\section{User-interface and Plumbing}

The entry point for the program is found in\footnote{``vcnc'' stands
for ``Virtual CNC.''}
\begin{verbatim}
vcnc.Main
\end{verbatim}
There's not a lot to say about this class: it merely kicks off the entire
program by creating a 
\begin{verbatim}
vcnc.MainWindow
\end{verbatim}
This is where all the plumbing comes together and is the primary
window for user activity. Most of what happens in the class is
standard stuff: menus, windows, \emph{etc}.

Alternatively, the {\tt main()} method can invoke a series of units
tests. Which of the two is done (ordinary GUI or unit test) depends on
which of two methods is called by {\tt main()}.\footnote{It wouldn't
be hard to the choice of which to run into a command-line argument,
but it seems like an unnecessary complication.} The unit tests are
discussed further, below, with the translator.

This window puts each file in a tabbed frame. By default, Java's {\tt
  JTabbedPane} doesn't have all the functionality desired, so there's
a fair amount of UI code in the {\tt vcnc.ui.TabbedPaneDnD} package, which is
discussed further, below.

\rulediv

In addition, each tab shown has a type that corresponds to the
contents of that tab. These types are managed by the {\tt
  vcnc.ui.TabMgmt} package. What is does is simple.

\begin{verbatim}
vncc.ui.TabMgmt.TabbedType
\end{verbatim}
is an {\tt enum} listing the possible types of tab contents and
\begin{verbatim}
vncc.ui.TabMgmt.TypedDisplayItem
\end{verbatim}
is an interface that each tab {\tt Component} must implement to
associate a type with its contents. Currently, all of these extend
{\tt JScrollPane}, but any number of other {\tt awt.Component}
objects would probably work as the base class.

Currently, there are only three of these types:
\begin{verbatim}
vncc.ui.TabMgmt.GInputTab
vncc.ui.TabMgmt.LexerTab
vncc.ui.TabMgmt.ParserTab
\end{verbatim}
The {\tt LexerTab} and {\tt ParserTab} are essentially identical,
\footnote{Clean that up.} and are used to display G-code output, which
is not editable. The {\tt GInputTab} is a bit more complicated because
this code is editable and due to toggling line numbers.

The final class found in this package is 
\begin{verbatim}
vncc.ui.TabMgmt.StaticWindow
\end{verbatim}
which is used to convert a tab to an independant window whose contents
can no longer be edited.\footnote{This is also similar to {\tt
  LexerTab} and {\tt ParserTab}.}

\rulediv

The remaining files in the {\tt vcnc} package are 

\begin{verbatim}
vcnc.TextInputDialog
vcnc.TableMenu
\end{verbatim}
The {\tt TextInputDialog} class is currently used to set the material
billet, but it's not being done in the way that it ultimately should
be done.

{\tt vcnc.TableMain} is leftover junk -- test code for {\tt
  JTables}. Get rid of it.

\rulediv

TALK ABOUT HOW THE MAIN WINDOW INVOKES THE UNDERLYING TRANSLATOR.

\subsection{{\tt vcnc.ui.TabbedPaneDnD} Package}

Two features beyond those provided by {\tt JTabbedPane}
are added: an ``X'' to close the tabs, and the ability to drag and
drop the tabs within each window or from one window to another. The
close-box is managed (mostly) through {\tt ButtonTabComponent}, and
the remaining classes manage drag \& drop, with {\tt TabbedpaneDnD}
being the main class of interest externally. See\footnote{I wrote this
ages ago, and it seems to work, but it's not pretty. I need to rewrite
it, make it more of a stand-alone thing that is useful more generally
and provide better documentation for it. It's messy Swing, so I'm
putting it off. }
\begin{verbatim}
vcnc.ui.TabbedPaneDnD.ButtonTabComponent
vcnc.ui.TabbedPaneDnD.GhostGlassPane
vcnc.ui.TabbedPaneDnD.TabbedPaneDnD
vcnc.ui.TabbedPaneDnD.TabDragGestureListener
vcnc.ui.TabbedPaneDnD.TabDragSourceListener
vcnc.ui.TabbedPaneDnD.TabDropTargetListener
vcnc.ui.TabbedPaneDnD.TabTransferable
vcnc.ui.TabbedPaneDnD.TabTransferPacket
\end{verbatim}

\subsection{Miscellany in the {\tt vcnc.util} Package}

\begin{verbatim}
vcnc.util.ChoiceDialogRadio
vcnc.util.ClickListener
vcnc.util.EmptyReadException
vcnc.util.FileIOUtil
vcnc.util.LoadOrSaveDialog
vcnc.util.StringUtil
\end{verbatim}

Not much to say about these...{\tt ClickListener} may be trash, and
they could all be tidied up with unused stuff taken out. They're
mostly older code pulled from other projects.

\begin{verbatim}
\end{verbatim}
\begin{verbatim}
\end{verbatim}
\begin{verbatim}
\end{verbatim}

\section{G-code Translator}

\ger\ is similar to a (very simple) compiler or interpreter. It
converts an input file of G-code to a simpler from. This
simplification happens as the code passes through a series of layers,
where each layer handles a particular aspect of the
simplification. The code related to this translation process is all in
{\tt vcnc.tpile.*}, either in that package or in a sub-package.

The lowest layer is the lexical analyizer (or ``lexer''). Because
G-code is so simple, with very little context-dependence, the lexer is
equally simple. It converts the incoming text file to a stream of {\tt
  Token} objects. Each token represents one of the letter codes (G, M,
I, J, \emph{etc}.) and any associated value. \ger\ allows the user to
extend ordinary G-code with user-defined functions, and these
functions are also converted to {\tt Token} objects.

The {\tt Token} objects are passed to the next layer, which is the
parser. The parser assembles the tokens into {\tt Statement}
objects. These statements correspond to the individual conceptual
steps of the program. The remaining layers convert these statements
into an increasingly stripped-down subset of the possible
G-codes, ultimately producing nothing but {\tt G00, G01, G02, G03}
codes for moving the cutter, plus a few M-codes and other codes that
pass through the process untouched.

The remaining layers are where the meat of the simplification
occurs. By layering the simplification process in this way, each layer
is made easier to understand, although it does lead to a certain
amount of repetative boilerplate. These layers are given numbers,
starting with {\tt 00}.

\subsection{\tt Lexer}

The code used for lexing is found in {\tt vcnc.tpile.lex}. Classes outside
this package -- the next layer of the translator -- need access to the
{\tt Lexer} and {\tt Token} classes. The other classes here have
default visibility, so are not accessible outside the package: there's
a token buffer and some internally used exception classes.

\subsection{\tt Parser}

The parser takes a stream of {\tt Token} objects, generated by {\tt
  Lexer}, and produces a series of {\tt Statement} objects. This code
is found in {\tt vcnc.tpile.parse}. Most of the classes in this
package extend {\tt StatmentData}. Each type of {\tt Statement} may
need different data, one class for ciruclar interpolation, one for
linear moves, \emph{etc}. The names for these classes start with
``{\tt Data}.''


\subsection{\tt Layer00}

This layer eliminates all calls to subroutines. In particular, {\tt
  M98} (call subprogram) and {\tt M99} (return from subprogram) are
replaced by the code of the corresponding subprogram(s).

IN ADDITION, this also eliminates certain items that serve no purpose
in a simulator, like {\tt M07, M09} and {\tt M09} for coolant
control, together with {\tt M40} and {\tt M41} for spindle high/low
and {\tt M48} and {\tt M49} for feed and speed overrides.

For several other commands, it's not clear what the appropriate action
should be, so they are treated as a ``halt.'' These codes are {\tt
  M00, M01} and {\tt M02} (various forms of ``stop''), along with {\tt
  M47} (repeat program).

BUG: I suspect that some of these should pass all the way through the
program, and only be dropped at the very end, when rendering.

\subsection{Unit Tests}

As noted above, the unit tests can be run from the {\tt main()}
method. The basic idea is that each layer of the translator, starting
with the lexer, is able to produce output as text (a {\tt
  String}). Each test is specified by an input file, which is run
through the translator up to a certain level. The output from this run
is compared to a static text file to see if they match.

\section{Geometry}

There are a few geometric problems that come up. 

\subsection{Arc Centers}

One problem is that we have two end-points of an arc and the radius,
and we need to know the center. This arises in the {\tt ArcCurve}
class; see {\tt calcCenter()}. Work in the $xy$-plane. Let the two
end-points be $(x_0,y_0)$ and $(x_1,y_1)$, with radius $r$, and the
center be at $(c_x,c_y)$. We want to determine the center given the
other values.

We have two equations and two unknowns:
\begin{eqnarray*}
  r^2 &=& (x_0-c_x)^2 + (y_0-c_y)^2 \\
  r^2 &=& (x_1-c_x)^2 + (y_1-c_y)^2
\end{eqnarray*}
Solve the first equation for $c_x$:
\begin{eqnarray*}
  (x_0-c_x)^2 &=& (y_0-c_y)^2 - r^2\\
  x_0-c_x &=& \pm\sqrt{(y_0-c_y)^2 - r^2}\\
  c_x &=& x_0 \mp\sqrt{(y_0-c_y)^2 - r^2}
\end{eqnarray*}
Substitute that into the second equation and solve for $c_y$:
\begin{eqnarray*}
  r^2 &=& (x_1-c_x)^2 + (y_1-c_y)^2 \\
  r^2 &=& \left(x_1-\left(x_0 \mp\sqrt{(y_0-c_y)^2 - r^2}\right)\right)^2 + (y_1-c_y)^2 \\
  r^2 &=& \left(x_1-x_0 \pm\sqrt{(y_0-c_y)^2 - r^2}\right)^2 + (y_1-c_y)^2 \\
  r^2 &=& (x_1-x_0)^2 \pm 2(x_1-x_0)\sqrt{(y_0-c_y)^2 - r^2} +
  (y_0-c_y)^2 - r^2 + (y_1-c_y)^2 
\end{eqnarray*}

Solving that for $c_y$ might work, but it's sort of
horrifying. Instead, use the fact that a line perpendicular to a chord
of the arc must pass through the center. That is, there is a chord
passing through $(x_0,y_0)$ and $(x_1,y_1)$. Let $(m_x,m_y)$ be the
mid-point of that chord. The slope of the chord is given by
$$s = {y_1-y_0\over x_1-x_0}$$
(and the order of the $x_i$ and $y_i$ doesn't matter). The slope of
the perpendicular to the chord is $-1/s$ or
$$s = {x_0-x_1\over y_1-y_0}$$
and the line perpendicular to the chord is given by
$$y - y_0 = \left({x_0-x_1\over y_1-y_0}\right) (x-x_0).$$
In particular, the center of the circle lies on this line, so we must
have
$$c_y - y_0 = \left({x_0-x_1\over y_1-y_0}\right) (c_x-x_0).$$

Let's simplify this mess...This is closer to what appears in the code. Define
\begin{eqnarray*}
  d_x &=& x_0-x_1\\
  d_y &=& y_0-y_1
\end{eqnarray*}
and
\begin{eqnarray*}
  m_x &=& (x_0+x_1)/2\\
  m_y &=& (y_0+y_1)/2.
\end{eqnarray*}
The slope of the chord is then $s = d_y/d_x$ and the slope of the
perpendicular is $-1/s$ or $-d_x/d_y$. The mid-point of the chord is
at $(m_x,m_y)$ so that the equation for the perpendicular to the chord
through the mid-point is given by
$$y - m_y = (-d_x/d_y) (x-m_x)$$
and we must have
$$c_y - m_y = (-d_x/d_y) (c_x-m_x)$$
or
$$c_y = m_y - (d_x/d_y) (c_x-m_x).$$
Substitute that into our first equation and get
\begin{eqnarray*}
  r^2 &=& (x_0-c_x)^2 + (y_0-c_y)^2 \\
  r^2 &=& (x_0-c_x)^2 + \left(y_0-\left[m_y - (d_x/d_y) (c_x-m_x)\right]\right)^2 \\
  r^2 &=& (x_0-c_x)^2 + \left[y_0-m_y + (d_x/d_y) (c_x-m_x)\right]^2.
\end{eqnarray*}
This is ugly, but it can be cleaned up. Substitute
$$u = c_x-m_x$$
and observe that $y_0-m_y = d_y/2$ and $x_0-m_x = d_x/2$. Then the
above can be written as
\begin{eqnarray*}
  r^2 &=& (x_0-c_x)^2 + \left[y_0-m_y + (d_x/d_y) (c_x-m_x)\right]^2 \\
  r^2 &=& (x_0-u-m_x)^2 + \left[d_y/2 + (d_x/d_y) u\right]^2 \\
  r^2 &=& (d_x/2-u)^2 + \left[d_y/2 + (d_x/d_y) u\right]^2 \\
  r^2 &=& (d_x/2)^2 -d_xu +u^2 + (d_y/2)^2 + d_xu + (d_x/d_y)^2 u^2 \\
  r^2 &=& (d_x/2)^2 + (d_y/2)^2  + (1+(d_x/d_y)^2) u^2.
\end{eqnarray*}
Set $q^2= d_x^2+d_y^2$ and the above becomes
\begin{eqnarray*}
  r^2 &=& (d_x/2)^2 + (d_y/2)^2  + (1+(d_x/d_y)^2) u^2 \\
  r^2 &=& q^2/4  + (q^2/d_y^2) u^2.
\end{eqnarray*}
Solve this for $u$:
$$u = \pm{d_y\over q}\sqrt{r^2-q^2/4}$$
and then
$$c_x = m_x \pm{d_y\over q}\sqrt{r^2-q^2/4}$$
and
$$c_y = m_y - {d_x\over d_y}(c_x-m_x) = m_y \mp{d_x\over q}\sqrt{r^2-q^2/4}.$$

\subsection{Arc Extent}

Given an arc move, we need to know the maximum extent of travel. That
is, the maximum and minimum values for $x$, $y$ and $z$. See {\tt
  ArcCurve.noteMaxMin()}. Assuming an arc in the $XY$-plane, it is
given in parametric terms as 
$$(x(t),y(t)) = (c_x,c_y) + r(\cos t,\sin t),$$
where $t$ ranges over some portion of the interval $[0,2\pi]$, and
this internval may ``wrap around,'' so it may be best to think of $t$
as being over an arbitrary interval. We want to know the maximum and
minimum values for $x$ and $y$; the value of $t$ where this happens
isn't important.

These extrema occur either at an end-point, or at a place where the
first derivative is zero. We have
$$(x^\prime(t),y^\prime(t)) = r(-\sin t,\cos t),$$
So the extrema in $x$ may be when $t$ is a multiple of $\pi$, and the
extrema in $y$ may be when $t$ is a an odd multiple of $\pi/2$ -- when
$t = (2k+1)\pi/2$, for $k\in\mathbb{Z}$.

\subsection{Arc Approximation}

Working with linear moves is easier than arcs, but if line segments
are used to approximate arcs, then it's important to have some measure
of the error. Consider the chord of a circle of radius $r$. Imagine
this chord being vertical through $(r,0)$.\footnote{Really need a
diagram.} Let $x$ be the distance from the mid-point of the chord and $e$
be the distance from the chord to the circle (so $e$ is the
error). You have two triangles around the $x$-axis, and let $y$ be the
height of one of these triangles. So the length of the chord is
$2y$. We have $r=x+e$ and $r^2=x^2+y^2$. Take $r$ as given, along with
$y$ as a particular length for the chord. We have
$$x = \sqrt{r^2-y^2}$$
so that
$$e = r -\sqrt{r^2-y^2}.$$
Rearrange:
\begin{eqnarray*}
  e &=& r -\sqrt{r^2-y^2}\\
  r-e &=& \sqrt{r^2-y^2}\\
  (r-e)^2 &=& r^2-y^2\\
  r^2-2re+e^2 &=& r^2-y^2\\
  e^2-2re + y^2 &=& 0,
\end{eqnarray*}
So that
$$e = {2r\pm\sqrt{4r^2-4y^2}\over2}.$$
DUH, same as before. Think in terms of the angle. Let the angle
subtended by the arc be $2\alpha$ so that $y=r\sin\alpha$. Then
$$e = r -\sqrt{r^2-r^2\sin^2\alpha} = r(1-\cos\alpha).$$
If we want $e$ to be no more than a certain size, then we must choose
$\alpha$ so that
$$\cos\alpha \geq 1-e/r.$$
For small $\alpha$, $\cos\alpha\approx 1-\alpha^2/2$ (Taylor series),
so
\begin{eqnarray*}
  1-\alpha^2/2 &\geq& 1-e/r \\
  \alpha^2 &\leq& 2e/r \\
  \alpha &\leq& \sqrt{2e/r}
\end{eqnarray*}
and $\alpha$ needs to be pretty darn small if $e$ is anything
reasonable. On the other hand, it's not that bad. If $r=1$, then
$2e/r$ around $0.010$ is pretty tight, which would give $\alpha =
0.1$. So you'd need $31.4$ chords around the circle. If $r=10$, then
the same estimates leads to 99 chords, which is still not crazy.
If $e$ is very tight, so that $2e/1$ is $0.0005$ ($e = 0.001$), then a
circle of radis 1 needs 140 chords and a circle of radius 10 needs 444
chords.

The question will be whether it's possible to save a significant
more calculation by working with linear moves or with arc moves.

\section{Rendering}

\subsection{Scale}

The biggest reason to include a concept of scale is for rendering: the
coarser the scale, the fewer positions to consider. In 3D, this works
by the cube of the change in scale, so this is a big factor. For
example, changing the scale from $0.001$~in to $0.005$~in will hardly
matter in terms of what the user sees, but the number of voxels in a
given volume goes down by a factor of 125. This can make the
difference between an imperceptable wait for a result and an
annoyingly long wait.

There is a tension between converting to a given scale earlier or
later. Rendering is done by breaking ``cuts'' into convex volumes, and
letting the coordinates that define these volumes remain at high
resolution until the last step of rendering, where it's determined
whether a voxel is inside or outside a given cut, is more
accurate. However, if arc moves are converted to linear moves, then
how that is done should be influenced by the scale.

As a compromise, non-linear moves are converted to a sequence of
linear moves at high resolution, and any coarsening of the scale is
delayed until the rendering step. In fact, there seems to be no
downside to expressing all moves at maximum accuracy and only
considering the scale when quantizing to voxels. In theory, it might
be possible to come up with some case where the linear moves used to
approximate an arc would be better chosen if the scale is taken into
account, but these cases seem odd and unlikely.


\subsection{Other Stuff}

In a lot of ways, rendering an image is the hardest
problem. Translation itself took time to implement because there were
certain decisions to make about coordinate systems and edge cases to
consider, but it was pretty clear and the algorithms aren't that
complicated. For 3D rendering, we are a lot closer to the edge of what
we have the horsepower to accomplish.

There are several ways to approach the problem if displaying some kind
of image that represents the part being cut.
\begin{enumerate}
\item Plain 2D, in black and white.
\item 2D with depth.
\item 2D with depth and surface normals.
\item 3D of some kind.
\end{enumerate}
Many of these could be done using either an ordinary array or a
quadtree/octree.

The plain 2D case is just an array (or quadtree) where the tool either
touches/cuts the material or it doesn't. Any time the tool descends
below $z=0$, there's a black pixel.

The ``2D with depth'' case is similar, but with something like a
z-buffer. For each pixel (whether that pixel is in an array or a
quadtree), there's a value for the maximum depth of cut seen over the
entire G-code program. Surface normals can be added by having two
arrays or quadtrees, one with the depth and one with these
normals. That takes considerably more memory -- at least four times as
much.

I'm not sure what the best way is to do the fully 3D case, but it
needs to be done if concave cutters or a 4th axis is to be dealt
with. I'm leaning toward octrees. I want something that doesn't
require a GPU to work, and I would prefer not to mess with a GPU at
all. OTOH, using a GPU might not be that difficult.

\subsection{Speed}

The simplest and most obvious way to deal with the 2D cases is to
think of the tool as cutting a certain circular pattern, and move this
pattern one pixel at a time and update the array over the entire
circle. Thus, if the cutter has radius $0.25$~in, then you have a
circle of area $0.125^2\pi$ or roughly 50,000 square thousandths. If
the cutter moves an inch, then that is 1,000 individual steps and
you'd have $50,000\times 1,000$ or $50,000,000$ individual positions to
check. More generally, if the tool has radius $r$ and it moves
distance $d$, both expressed in terms of the rendering resolution
(like thousandths), then you have to consider $\pi r^2d$ individual
positions/voxels.

The above is too much. The cutter's total distance of travel will
typically be large (many inches). Let $u$ be the scale or rendering
resolution, expressed in multiples of one thousandth (or whatever the
basic unit of the machine may be). Let $t$ be the total distance
travelled by the cutter over an entire program, in thousandths. Then
we're looking at $\pi (r/u)^2 (t/u)$. Assume $r=2^8$ (roughly a
half-inch cutter) and $t$ could be $2^{10}$ inches or $2^{20}$ total
units travelled. So, overall, we have something like $2^{38}/u^3$
voxels to visit. Obviously, letting $u$ be something like 4 or 8 would
help a lot, but it's still too much.

This can be reduced by adding a slight complication. Instead
of considering all the voxesl of the cutter at each position, consider
only those voxels on the leading edge of travel. For an ordinary
end-mill, this changes from requiring that the area of the cutter be
considered to only the perimeter. The savings is less for something
like a ball-mill or v-cutter; in those cases, the ``leading edge'' is
really a leading surface, so the reduction is only by a factor of
about one half. Then again, you could let the outer perimeter cut
(like the semicircle in the vertical plane of a ball-mill), and at the
two ends of the cut you would have to consider the hemispheres. So the
savings is almost as large.




\end{document}
