
% Discussion of different way to go about writing a G-code simulator.

\documentclass[titlepage,oneside,10pt]{article}

% Need this for pmatrix (among other things).
\usepackage{amsmath}

% Needed to include graphics objects (from metapost).
\usepackage{graphics}

\usepackage{epic,eepic} \setlength\maxovaldiam{60pt}

% So that I don't have to type this all the time.
\newcommand{\ger}{\textsc{Ger}}


% For ``long'' tables.
\usepackage{longtable}

\begin{document}

\raggedbottom

% These are for marginal notes on the right or left of the main text.
\newcommand{\mymargin}[1]{\marginpar{\rm\tiny #1}}
%\newcommand{\mymargin}[1]{\marginpar{\rm\scriptsize #1}}
\newcommand{\leftmar}[1]{\reversemarginpar \mymargin{#1}}

\section{Yet to Do}

There's a long list of things wrong with this. It's semi-usable, but
here's a list of things to revisit (or visit).
\begin{itemize}
\item Allow user to specify font size (and maybe the font, although
  any sane person will want a mono-spaced font).
\item The basic framework for wizard code is there, but no more than
  the framework. In particular, the means of outputing more types of
  {\tt Statement} needs to be fleshed out.
\item Certain wizards (like Bezier curves) should have a UI specific
  to them. Other UIs might be handy too (bolt circles), but are less
  necessary. None of these wizards exist yet in any form.
\item The ``machine setup'' wizard framework is also minimal. This is
  for things like specifying the tool table, work offsets table,
  billet dimensions, etc.
\item Go through the code and explain (in this document) how each
  transformation takes place. There are edge cases that might surprise
  the user. Some of the more detailed calculations belong in the
  developer's manual.
\item There is no 3D rendering at all.
\end{itemize}

\section{G-Code}

There is no universally accepted standard for ``G-code.'' The
most commonly used codes do adhere to common usage, but usage for the
less common codes varies from one CNC controller to the next. \ger\ 
assumes that the common codes do follow a certain form; for 
other codes, how they are dealt with is left up to the user.

\ger\ is more flexible than many real-world controllers about the order
and grouping of codes. For instance, something like this is accepted:
\begin{verbatim}
G01 X1.0 G21 G01 Y250.0
\end{verbatim}
and would be interpreted as
\begin{verbatim}
G01 X1.0 
G21 
G01 Y250.0
\end{verbatim}
Statements may be terminated by either a new-line (\emph{i.e.}, a
``carriage return'' or ``enter''), or by a semi-colon. Thus,
\begin{verbatim}
G01 X1.0; Y2.0
\end{verbatim}
is valid and means something entirely different than 
\begin{verbatim}
G01 X1.0 Y2.0
\end{verbatim}
If it is possible for \emph{you} to make sense of your program by 
reading the codes one chunk at a time, then the simulator can
probably make sense of it too, although the controller of an actual
machine may be more strict or may have a different interpretation.

Which leads to the next point. \ger\ is both a simulator and a
translator. As a simulator, it can be used to view a 3D rendering of
the part generated by the G-code, or it can be used to translate complex or
idiosyncratic code to something simpler that will be accepted by a
wide variety of machines. For example, not every controller accepts
{\tt M98} (call sub-program), even though it is fairly
standard. \ger\ accepts {\tt M98} and converts it to more basic
commands that any real-world machine will accept.

For uncommon codes, \ger\ assumes that the code extends to the end of
the line on which it appears (or to a semi-colon). \ger\ can't know
the idiosyncracies of every machine, so these uncommon codes simply
pass through the translator so that they can be fed to a physical
machine that may understand them. By default, the \ger\ simulator ignores
these codes. For something like {\tt M08} or {\tt M09} (coolant
on/off), ignoring the code is a natural choice; but in other cases,
like {\tt G84.3} (a left-handed tapping cycle), it is possible to
define the code's meaning so that \ger\ can simulate it
faithfully. See SOME-FUTURE-SECTION.\footnote{In fact, I've pretty
much taken out the stuff that might be used to do this. In theory, it
could be put back, but I don't see this as likely to be much
desired. \ger\ either knows what a code does or it rejects the code.}

In addition to the ability to define uncommon codes, it is also
possible to define your own functions that act as ``wizards.'' For
example, one might define
\begin{verbatim}
Pocket 75.0 30.0 10.0 
\end{verbatim}
to cut a pocket 10mm deep, measuring 75mm by 30mm. How to define these
wizards will be discussed below (WHERE?). In particular, wizard
commands must start with two consecutive letters.

\section{Common Codes}

The codes described below are common enough to assume that they take a
standard form and have a standard meeaning. For any codes not on this
list, \ger\ assumes that the full statement runs to the end-of-line or
the next semi-colon.

The table below is organized by ``statement,'' meaning a series of
inputs that makes sense together. To make the descriptions more concise, {\tt
  [...]} is used for an optional term, \emph{n} means a number,
possibly with a decimal, and \emph{w} means a whole number. What
appears below is a quick summary; the individual codes are described
in more detail later.

\vskip 0.25cm
\begin{longtable}{lp{10cm}}
\em{Move} & \tt{[X}\em{n}\tt{]} \tt{[Y}\em{n}\tt{]}
\tt{[Z}\em{n}\tt{]} \tt{[F}\em{n}\tt{]} \\ 
& Notice the use of {\tt [} and {\tt ]}: each of these terms is
optional. \\
& This command moves the tool along a line segment to the given X, Y,
Z, at the given feed rate. When in rapid mode ({\tt G00}), {\tt F} is
not allowed. Once a feed rate is given, it will be used in {\tt
  G01}-mode until the feed rate is changed.\\
{\tt G00}&Rapid travel mode. The program begins in rapid travel
mode. If the very first line is {\tt X1.000 Y1.000}, then the tool
will move to that position in rapid mode.\\
{\tt G01}&Ordinary (not rapid) travel mode.\\
{\tt G02}& Clockwise arc.\\
{\tt G03}&Counter-clockwise arc. \\
& {\tt G02/G03} {\tt [X}\em{n}\tt{]} \tt{[Y}\em{n}\tt{]}
\tt{[Z}\em{n}\tt{]} {\tt [I}\em{n}\tt{]} \tt{[J}\em{n}\tt{]}
\tt{[K}\em{n}\tt{]} \tt{[F}\em{n}{\tt ]} \emph{or} \\
&{\tt G02/G03} {\tt [X}\em{n}\tt{]} \tt{[Y}\em{n}\tt{]} \tt{[Z}\em{n}\tt{]}
\tt{[R}\em{n}\tt{]} \tt{[F}\em{n}\tt{]}\\
&Circular interpolation. If an arc is specified using {\tt R} rather than {\tt
  I/J/K}, then a positive {\tt R}-value produces an arc that subtends
less than $180^\circ$; using a negative {\tt R}-value produces an arc
that's more than $180^\circ$.\\
&Also, various machines may treat the ZX plane slightly
differently, depending on whether they are ``standard'' or ``vertical
machining.'' So, {\tt G02} and {\tt G03} may be treated in opposite
fashion. Basically, what does ``clockwise'' mean? See Smid, \emph{CNC
  Programming Handbook}, p. 280.\\
&To cut a complete circle, the X/Y/Z values given with the command
must be equal to the starting point of the tool, and you must use
the I/J/K format, not an R-value.\\
&Geometrically, the distance of the center to each of the two
end-points must be the same (that distance is the radius). When using
the I/J/K format, it's not unusual for the center of an arc to require
many digits to specify exactly, perhaps even an infinite number. So
that the programmer only needs to input a reasonable number of digits,
a certain amount of leeway must be allowed. Ger will accept a center
if the distances to the two end-points are within 1\% of each other,
and it will assume that the radius of the circle is equal to the
larger of these two distances.\\
{\tt G15}&Polar coordinates off.\\
{\tt G16}&Polar coordinates on. \\
&You may enter polar coordinate mode while in incremental mode
({\tt G91}) or you may enter incremental mode after invoking {\tt
  G16}, or go back to absolute mode ({\tt G90}). Not every real-world CNC
controller allows this. Ger also allows circular interpolation ({\tt
  G02/G03}) while in polar coordinate mode, another thing that most
real-world machines do not permit. Use of {\tt G01} or {\tt G00} while
in polar coordinate mode is \emph{not} allowed.\\ 
{\tt G17}& Work in the XY-plane (the default).\\
{\tt G18}& Work in the ZX-plane.\\
{\tt G19}& Work in the YZ-plane.\\
& While in cutter comp mode, only the standard XY-plane may be
used. This is typical for real-world controllers.\\
{\tt G20}&Inches.\\
{\tt G21}&Millimeters.\\
& Ger works internally using either inches of millimeters, but you can
freely change from one unit of measure to the other within your
G-code. In fact, Ger is more flexible than many real-world
controllers; you can sprinkle {\tt G20}'s and {\tt G21}'s all over the
place. The only (?) reason the internal setting for \ger\ matters is
how it interprets things like tool tables. \\  
{\tt G28}&Machine Zero Return. This is accepted, but everyting from
{\tt G28} to the end of the line (or a semi-colon) is ignored. On the
simulated machine, it has the effect of moving the tool to the
position of the tool turret.\footnote{In fact, I'm not sure whether
  I've implemented this.}\\
{\tt G40}&Cancel cutter comp, also known as ``tool radius compensation.''\\
{\tt G41}&Cutter comp, left.\\
{\tt G42}&Cutter comp, right. \\
&{\tt G41/G42 D}\emph{w} \emph{or}\\
&{\tt G41/G42 H}\emph{w} \\
& The cutter comp is given relative to a value stored in a
D-register or an H-register. The D-register typically contains the
diameter of the tool and the H-register contains a value for tool
wear. These are provided by using the ``Tool Turret'' dialog. As a
reminder, the left/right distinction here refers to the 
position of the tool relative to its path. Thus, {\tt G41} puts the
tool to the left of the path it follows.\\
&Ger does things slightly differently than how many real-world
controllers seem to work. As soon as a {\tt G41/42} appears, the
virtual machine peeks ahead to see what the next statement will be and
it bumps the tool out by the tool radius to be ready to cut the very
first curve with the correct cutter comp. I think this is better than the way
it's typically done, but it could confuse people who are trying to
test against a particular real-world machine. There seems to
be so much variation from one controller to another in how this aspect of
cutter comp is handled, that some degree of mis-match between Ger and
the real world is unavoidable. In any case, the usual advice applies
to \emph{all} machines: invoke cutter comp away from the part and make
at least one move before contacting the part. \\
& Note that Ger does not allow things like\\
& {\tt G41 X\#\# Y\#\# Z\#\# D\#\#}\\
& You can't have any extra stuff between the {\tt G41} and the {\tt D}
or {\tt H}-register specification. Many real-world CNC machines do
allow this.\\
& One last point about cutter comp. Ger allows cutter comp to be used
on interior angles, while many real-world controllers do not. However,
Ger is not as smart as some real-world controllers about how it deals
with this case. For instance, if the tool path is specified to form a
very acute vertex, say $5^\circ$, and the tool makes many small moves (less
than the tool radius) near this vertex, then the tool path in cutter
comp mode might not be what you expect. On the other hand, it's hard
to imagine what a person might intend in a case like this, so it seems
better to let the tool do oddball things as a way of informing him
that he \emph{input} something oddball. \\
& About the only sitation I
can imagine where this would arise is if you feed Ger code that was
posted by a (rather stupid) CAM program. The way to ``fix'' this is to
have infinite look-ahead in cutter comp mode so that Ger can remove
all of those small moves near the apex from the statement stream, but
it's not clear to me that this is something that \emph{should} be
fixed. It seems better for the controller to behave in a way that
makes it apparent that the input G-code is weird, and probably wrong, rather
than acting in a way that hides from the user potential problems with his
G-code. \\
{\tt G43}&Positive tool length offset (TLO).\\
{\tt G44}&Negative tool length offset.\\
{\tt G49}&Cancel tool length offset.\\
&{\tt G43/G44 [Z}\emph{n}{\tt ] H}\emph{w}\\
& The simulator considers the {\tt Z}-value to be
optional, although it is required by many real-world machines. \\
& As a reminder, in the case of {\tt G43}, this has the effect
of adding the value in the H-register to all Z-moves until a {\tt G49}
is reached. If the H-value is negative, then the tool will cut more
deeply. {\tt G44} has the opposite effect.\\
&A program may not enter TLO while in incremental mode, when using
polar coordinates, or while in cutter comp mode. However, once TLO is
in effect, a program may enter any of these modes. If your program
does enter one of these modes, then it must leave these modes before
cancelling TLO. \\
{\tt G52}&Local coordinate system.\\
& {\tt G52 [X}\em{n}\tt{]} \tt{[Y}\em{n}\tt{]} \tt{[Z}\em{n}\tt{]} \\
& This changes the origin to be the point at the given X, Y, Z, stated
relative to the PRZ. All moves after {\tt G52} are as though the PRZ
is at this new position. Each of the X, Y, and Z values is optional;
omitted values are assumed to be zero (no change). This is cancelled
with {\tt G54}. \\
{\tt G54}& Return to normal work offset. That is, use the PRZ as the
origin. If a tool length offset is in force ({\tt G43} or {\tt G44}),
then it remains in force after calling {\tt G54}.\\
{\tt G55}&\\
{\tt G56}&\\
{\tt G57}&\\
{\tt G58}&\\
{\tt G59}&Codes {\tt G54} through {\tt G59} are similar to {\tt G52},
but the values used for X, Y and Z come from the ``work offsets''
table. So, these codes appear alone, simply as {\tt G55} (say),
without any additional codes.\\
&Just as with TLO mode, none of these codes, including {\tt G52}, may be
entered or exited while in incremental mode, polar coordinate mode, or
cutter comp mode.\\
&Normally, the values used for work offsets are the distance from
machine zero to program zero (the PRZ), but the virtual machine has no
``machine zero.'' For the purpose of work offset settings, think of
the PRZ as the machine zero. Put another way, the work offset values
should be the amount in each direction, X, Y and Z, that you want the
tool to be offset from the position it would have without the change
due to applying one of {\tt G55} through {\tt G59}.\\
{\tt G90}&Absolute mode.\\
{\tt G91}&Incremental mode.
\end{longtable}
\vskip 0.50cm
As you can see, none of the canned cycles, like {\tt G81} for
drilling, have been implemented. Some of them might manage to get
through my system without causing an error, but none of them do
anything. I haven't thought very hard about this, but I don't think that
they would be hard to add.

Here are the valid M-codes.
\vskip 0.25cm
\begin{longtable}{lp{10cm}}
{\tt M00}&\\
{\tt M01}&\\
{\tt M02}&These are different forms of ``program stop.'' The simulator
simply halts when it reaches one of these.\\
{\tt M03}&Spindle on, clockwise.\\
&         {\tt M03 S}\emph{w}\\
& where \emph{w} is the spindle speed.\\
{\tt M04}&Spindle on, counter-clockwise. Similar to {\tt M03}. The
program notes internally that the spindle is on, but it has no real
effect.\\
{\tt M05}&Spindle off.\\
{\tt M06}&Tool change.\\
& {\tt M06 T}\emph{w}\\
& changes to the tool in the \emph{w} position of the turret. The
simulator moves the tool in a straight line from its curent position
to the tool turret, changes to the new tool, and stops at that
position.\footnote{These motions don't appear in the translator part of the
  program; they are used to detect tool crashes.} \\
{\tt M07}&\\
{\tt M08}&\\
{\tt M09}&These coolant on/off commands are accepted, but they do
nothing.\\
{\tt M30}&Halts the program.\\
{\tt M40}&\\
{\tt M41}&Spindle high/low. Accepted, but ignored.\\
{\tt M47}&Repeat program. The simulator halts immediately if it
reaches this code.\\
{\tt M48}&\\
{\tt M49}&Feed \& speed override on/off. Accepted, but ignored.\\
{\tt M98}&Call subprogram.\\
&{\tt M98 P}\emph{w} {\tt L\emph{w}}\\
& This calls a sub-program, where {\tt P} gives its number, and {\tt
  L} gives the number of times to call it.\\
{\tt M99}&Return from subprogram.
\end{longtable}

Two other codes are accepted: the {\tt O} code for a program number,
and {\tt N} may be used for line numbers. Any line numbers given with
{\tt N} are ignored. Every program is required to start with an O-value.

If there are any errors in the program, then the line number on
which the errors are detected are given by counting lines in the input
text file, not with reference to any {\tt N}-specified line numbers in
the input G-code. Thus, if an error is reported on line 3 (say), then the
problem was detected on the third line from the top, not the line that
starts with {\tt N3} (assuming there is such a line).

Finally, comments are expressed with parenthesis, {\tt (...)}, as
usual. The simulator accepts multi-line comments, but does not accept
nested comments.


\vskip 2cm

\section{Transpiling}

Ger woks by transpiling (really, simplifying) the input code to
the simplest possible subset of valid G-codes. This is done through a
series of layers. When errors occur, it may be helpful to
understand how the process of code simplification happens.

The first step is the lexical analizer, or ``lexer''. The lexer
converts the input text to a sequence of ``tokens.'' These tokens are
of one of a few types. Ordinary G-code appears as a combination of
letters, optionally followed by numbers. Examples include {\tt G0},
{\tt M30}, {X3.205} and the like. The second type of token is
associated with externally defined wizards. These include the
function name, like {\tt MyFunction}, and any arguments to such
functions. These areguments are limited to numbers and strings
delimited by a pair of double-quote (\ {\tt "}\ ) characters. The
arguments to a wizard are space-delimited, and it is assumed that
the arguments run to the end of the line on which the wizard is
called.

The next layer is the parser. It converts the output from the lexer to
syntacially valid statements. 

\section{Coordinate System}

The coordinate system is inherently tied up with the
particular physical machine on which the G-code is intended to
run.\footnote{In principle, it would be possible to specify the machine
geometry sufficiently, including any tool turret, so as to detect
tool crashes. But there are too many possible machine
arrangements, and for this to work well, it would also be necessary to
specify any jigs and fixtures.} Each time a program starts, the tool
is assumed to be located at $(X,Y,Z) = (0,0,0)$, and the machine
defaults to working in inches. It's often unnecessary to consider the
coordinate system in any further detail, but there are certain
additional settings and controls available.


These codes are particularly related to
the coordinate system:
\begin{verbatim}
G20/21 (inches/mm)
G28 (machine zero return)
G52/54 (local coordinate system/cancel)
G55/56/57/58/59 (use work offsets table)
G92 (I MIGHT WANT TO ADD THIS ONE).
\end{verbatim}

\subsection{Machine Zero and Work Offsets}

The location of ``machine home'' or ``machine zero'' is the position
of the tool at the start of the program; {\tt G28} always returns the
tool to this position, in rapid mode. The PRZ (``program reference
zero,'' or ``part reference zero'') starts off the same as machine 
zero, but can be adjusted with {\tt G52} or {\tt G55-G59}. Absolute
coordinates are always given relative to the current PRZ.

{\tt G52} is given with values for {\tt X, Y} and {\tt Z}, and each of
these values defaults to zero if it is omitted. The location of the
PRZ moves to the given {\tt (X,Y,Z)} value relative to machine zero
(which is fixed for all time). For example, after
\begin{verbatim}
G52 X1.000 Y2.000 Z0.500
\end{verbatim}
appears, the location of {\tt (0,0,0)} will be the same as {\tt
  (1,2,0.5)} had been prior to the {\tt G52}. Put another way,
subtract the values given to {\tt G52} from all coordinates, and you
have the location relative to machine zero.

The codes {\tt G55} through {\tt G59} works just as {\tt G52}, except
that the {\tt X, Y} and {\tt Z} values are taken from the ``work
offsets table.'' This table is shared between all programs, and it is
also possible to modify this table with the {\tt SetWorkOffset}
command before the {\tt O}-command with the program number.

The PRZ is reset to machine zero with {\tt G54}.



\vskip 1cm



Blah, blah. Should probably just include most of the discussion in the
long able above, with a few remarks here.

NOTE: It is tempting to simply get rid of the workoffset table and the
commands related to it ({\tt G55}, etc.). I don't think people really
use this.

The bottom line is that whenever a program starts, it assumed that the
tool starts off at $(0,0,0)$. This is the location to which {\tt G28}
returns. You can also use {\tt G52/5/6/7/8/9} to change this and all
of these are canceled by {\tt G54}.

For the work offset tables, want there to be a table that could be
used for every program, but it also makes sense to provide some
special commands that can be used before the program starts to set
these table values.

It also means that -- if I want to specify the billet at all -- then
it needs to be done relative to the initial origin. The three ranges
would be given in something like $X=[a,b]$, $Y=[c,d]$ and $Z=[e,f]$,
where these values could be negative.


Exactly how to define the geometry of the machine and the part,
together with various concepts of ``origin,'' is a difficult
question. On the other hand, it's often the case that certain obvious
defaults will suffice. To fully define the geometry of the machine,
follow these steps:
\begin{enumerate}
  \item Define the size of the tool table.
  \item Set the location of the tool turret relative to the lower-left
    corner of the tool table. The turret location is the position to
    which the spindle will return for a tool change.
  \item Define the dimensions of the billet of material. By default,
    it is assumed to be centered on the tool table.
  \item Set the PRZ relative to the billet. By default, this is taken
    to be at the lower-left corner of the billet.
\end{enumerate}  
The series of steps above is fussy and not necessary in most
cases. Usually, it suffices to work with default settings. In this
case, the billet is taken to be just large enough for the cuts that
appear in the G-code (should allow setting some kind of margin) and
the PRZ ends up being at the lower-left of the material. Tool changes
happen in-place, without a move to a particular location for the tool
turret. Things like wook tables might be trickier though.

THIS WON'T WORK. For one thing, the tool table moves on some
machines. Other machines might have the tool table move in only X or
Y. It's just too fiddly. What I need to do is break the list of codes
up into the obvious ones, and the ones that are somehow sensitive to
the machine geometry. The commands that are somehow sensitive to the
geometry are {\tt G28} (machine zero return), {\tt G43, G44, G49} (to
do with tool length offset), {\tt G52-G59} (work offsets), {\tt M06}
(tool change)

For things that have to so with the coordiante system, it's reasonable
to make everything relative to a single PRZ, which can default to the
lower-left corner of the billet, perhaps with some margin.

I will ignore the location of the tool changer, and just have the tool
magically change in place. The user could redefine {\tt M06} somehow
though to have it move to a certain location.

That leaves the tool length offset commands, and those aren't a big
deal.

Overall, the best approach seem to be to specify all of this stuff
within the G-code file. Each program could have different settings,
and it's misleading to use the main window to do this. What could be
done in the main window, as a kind of machine-wide setting, is the
tool table. In any case, this is a more complicated thing that could
be difficult to do with simple text.

So, what's needed is the following.

The first thing is to set the internal units used by the simulated
machine. Say
\begin{verbatim}
SetUnits ["inch"|"mm"]
\end{verbatim}
The default is {\tt "inch"}. This indicates the units for things like
part dimensions and work offsets.

Once the units have been set, there are two ways to determine the part
dimensions and machine zero (PRZ). The simplest is to say
\begin{verbatim}
SetMargin <value>
\end{verbatim}
where {\tt <value>} is the marginal amount that appears beyond the
extent of where the tool cuts. In this case, the tool is assumed to
touch the billet when the $z$-position drops below zero.

Another way to set the part dimension is to first say
\begin{verbatim}
SetPart <x> <y> <z>
\end{verbatim}
where the values are the size of the raw (cubic) billet in the three
dimensions. By default, machine zero (PRZ) is the top-lower-left of
this material. The PRZ can be changed by saying
\begin{verbatim}
SetZero <x> <y> <z>
\end{verbatim}
where the values are relative to the top-lower-left corner. For example,
\begin{verbatim}
SetPart 4.0 2.0 1.0
\end{verbatim}
means that the part is $4\times2\times1$ ($x$ by $y$ by $z$). Now 
\begin{verbatim}
SetZero 0.25 0.25 0.10
\end{verbatim}
means that the PRZ is $0.25$ in from the lower-left corner in both
directions, and lowering the tool to Z$=0$ cuts off the top $0.10$ of
the billet. 



\section{Wizards}

THIS IS MORE OF A PLACEHOLDER...

The easiest way to develop a new wizard is through the
command-line. Say (something like)
\begin{verbatim}
ger -compile mywizard.java
\end{verbatim}
The only reason to do this, and not use {\tt javac} is to avoid
understanding the Java development framework any more than absolutely
necessary.

You might also say
\begin{verbatim}
ger -translate gcode.txt
\end{verbatim}
where {\tt gcode.txt} uses {\tt mywizard}. The issue here is that
setting things up to allow \ger\ to permit interactive Java
development is hard for at least two reasons. First, is simply the
UI -- \ger\ can't be an IDE. Second, is that it is hard to reload a
given {\tt .class} files. As the user changes his {\tt .java} file,
the {\tt .class} file will change, and it's hard to reload changes to
a class. The JVM doesn't understand that the current {\tt MyWizard}
class is different than the {\tt MyWizard} it loaded earlier. It's
possible to do this (IDEs do it), but a lot of messing around.


\vskip 2cm





\part{Old Java Version}

\section{System Overview}

When the program is launched a window comes up with several menu
choices. The ``File'' menu acts as you would expect. You can have
multiple G-code programs open at the same time; each one appears in
its own tab. You can edit and save programs from here too, although
the editor is nothing fancy.

The ``Render'' menu generates a picture of the object described by
the G-code program. There are two choices: ``2D Display'' and ``3D
Display''. Either choice generates a picture for the G-code in 
the foremost tab. The picture will appear in a new window, and 
the image is not automatically updated when you change the
G-code. If you modify the G-code, and you want to see the result,
choose ``Render'' again. Images for both versions of the G-code (new
and old) will remain open. These images (and the data used to generate
them) take a lot of memory. If you have several of these image windows
open, then you could easily run out of memory.

Choosing ``2D Display'' shows a static image of the object, looking
from above with each point shaded by the depth of cut. There is no
user interaction (although you can make the window larger or
smaller). If you choose ``3D Display'', then you get an interactive
3D image of the object. You can't use the mouse to rotate the
image. Instead, use J and L to rotate left and right, I and M to
rotate up and down, and U and O to twist the object in place. Use A
and Z to zoom in and out. Also, X will close the window. Upper and
lower case letters both work.

The ``Settings'' menu has several choices. Most of the choices should
be self-explanatory, but not everything works. Both ``Scale'' and
``Inches/mm'' do work. If memory is tight, or if the
image updates uncomfortably slowly, then reducing the scale
should help. If the scale is $s$, then the virtual machine should work
exactly as though the tool moves in steps of $1/s$ inches (or
millimeters). Put another way, it's as though you multiplied all of
the coordinates in your program by a constant. For instance, if you
change the scale from $1,000$ to 100, that's like multiplying all
coordinates by $0.10$. For objects that are only a few inches in size and are
of modest complexity, leaving the scale at 1000.0 should be fine, but
for large objects, you may need to make this smaller. In fact, unless you're
doing something really intricate, a value of 100 (or even less) is
probably fine. 

The ``Work Offsets'' dialog does work (although it hasn't been very
thoroughly tested), and it should be clear what to do. 

Don't try to use the ``Uncut Shape'' dialog. It's meant as a way for
the user to indicate the size of the raw material, but it doesn't
work. Right now, the program will automatically use an object that's
large enough to encompass all cuts made by the program.

The ``Tool Table'' works, but only up to a point. The system
recognizes $D$ and $H$ values from this table when using cutter comp
and tool length offsets, and it knows the diameter of the tool. What
it can't do is use anything other than simple endmills. If you change
the diameter value in the $D$-column, then you must hit the return/enter
key to tell dialog that you mean it before you click the ``OK''
button. One last thing: there's a secret default tool. This tool has a
cutting diameter of $0.020$ inches, but the cutter comp is $0.250$
inches. This is handy sometimes when you're trying to understand what
cutter comp is doing; it makes it easier to visualize the actual path
of the tool. If there's no {\tt M06} in your program, then you'll get
the secret default tool. 

Concerning Java and memory...When you launch the program that I wrote,
your computer isn't really running my program. The computer runs Java,
which in turn opens up and runs my program. This means that, whenever
my program needs more memory, it must ask the Java system for the
memory it needs. You may have oodles of memory on your machine, but
Java won't let my program use it unless you tell Java that it's OK. I
provided one way around this problem (a {\tt .bat} file), but there
are others.

\section{G-codes and M-codes}

I tried to implement all of the G-codes that one might care about and
that are likely to be consistent across different makes of
controller. Here's a list, with some comments.
\vskip 0.25cm
\begin{longtable}{lp{10cm}}
{\tt G00}&Rapid travel.\\
{\tt G01}&Ordinary move. There are no surprises with these
two. However, for all practical purposes, my program makes no
distinction between {\tt G00} and {\tt G01}. The program does not
check for things like whether the spindle is  on when you cut, or
whether you're trying to cut in rapid mode. Likewise, the program
accepts a feed rate (F-value), but the feed rate has no effect on the
output. \\
{\tt G02}&\\
{\tt G03}&Circular interpolation. These do what you'd expect,
although I'm not very confident that they work correctly if you are not
in {\tt G17} mode.\\
{\tt G04}&Dwell. This is accepted, but ignored. Actually, it might
cause an error, depending on how you use it. I'm not sure.\\
{\tt G15}&Polar coordinates off.\\
{\tt G16}&Polar coordinates on. Again, using {\tt G18} and {\tt G19}
might not work right. I tried to make them work correctly, but it
hasn't been thoroughly tested.\\
{\tt G17}&\\
{\tt G18}&\\
{\tt G19}&See above and below for caveats about {\tt G18} and {\tt G19}.\\
{\tt G20}&Inches.\\
{\tt G21}&Millimeters. The abstract machine works internally in either
inches of millimeters, but you can freely change from one unit of
measure to the other within your G-code. In fact, my program is
probably more flexible than many real controllers, perhaps dangerously
so. You can sprinkle {\tt G20}'s and {\tt G21}'s all over the place.\\ 
{\tt G28}&This is accepted, but ignored. Smid calls it ``machine zero
return''; I'm not sure exactly what it should do.\\
{\tt G40}&Cancel cutter comp.\\
{\tt G41}&Cutter comp, left.\\
{\tt G42}&Cutter comp, right. These work as they should, but (as
usual) they might be wrong under {\tt G18} and {\tt G19}. Also, I do
things differently than how most real CNC machines seem to
work. As soon as a {\tt G41/42} appears, the virtual machine peeks
ahead to see what the next statement will be, and it bumps the tool
out by the cutter comp amount to be ready to cut the very first curve
with the correct cutter comp. I think this is better than the way
it's typically done, but it could confuse people who are trying to
test against a particular real-world machine. Anyway, there seems to
be so much variation 
from one machine to another in how this aspect of cutter comp is
handled, that some degree of mismatch between my virtual machine and
the real world is unavoidable. 
\vskip 0.10cm
Dealing with cutter comp is messy and it's the thing most likely to
have bugs. Also I chose to handle cutter comp by using what are
probably very unorthodox methods. They are powerful, in the sense that
I could extend my program to handle cutter comp on (as yet
unspecified) curves that are more complex than line segments and arcs
of circles, but there could be bugs for reasons
that are hard to track down.\\
{\tt G43}&Tool length offset, in the normal case.\\
{\tt G44}&Tool length offset, in the negative sense. I doubt that
this is used often, but it works.\\
{\tt G49}&Cancel tool length offset. Tool length offset works as it
should. One oddity is due to the fact that, in a virtual machine,
tools don't have a natural length. For the time being, treat all tools
as though the machine magically knows where the tip of every tool
is. If you use {\tt G43} after setting $H=0$ in the tool table, then
it will act like it should. If $H>0$, then the tool will cut more
deeply under {\tt G43} and less deeply under {\tt G44}.\\
{\tt G52}&Allows you to specify a new local reference frame. This is
something like a work offset ({\tt G54},$\ldots$,{\tt G59}), but the
coordinates are specified in the program, and must be given relative
to the PRZ.\\
{\tt G54}&\\
{\tt G55}&\\
{\tt G56}&\\
{\tt G57}&\\
{\tt G58}&\\
{\tt G59}&To change work offsets. These work as you would expect, but
the coordinates must be given relative to the PRZ in the work offset
table. There's no natural way to define ``machine coordinates'' other
than the PRZ.\\
{\tt G90}&Absolute mode.\\
{\tt G91}&Incremental mode. These work as expected.
\end{longtable}
\vskip 0.50cm
As you can see, none of the canned cycles, like {\tt G81} for
drilling, have been implemented. Some of them might manage to get
through my system without causing an error, but none of them do
anything. I haven't thought very hard about this, but I don't think that
they would be hard to add.

Here are the valid M-codes.
\vskip 0.25cm
\begin{longtable}{lp{10cm}}
{\tt M01}&Smid calls this ``compulsory program stop.''\\
{\tt M02}&And he calls this ``optional program stop.'' In both cases,
I make the program halt, like an {\tt M30}.\\
{\tt M03}&Spindle on, clockwise.\\
{\tt M04}&Spindle on, counter-clockwise. The program notes internally
that the spindle is on, but it has no real effect.\\
{\tt M05}&Spindle off.\\
{\tt M06}&Tool change.\\
{\tt M07}&\\
{\tt M08}&\\
{\tt M09}&These coolant on/off commands are accepted, but they do
nothing.\\
{\tt M30}&Halts the program.\\
{\tt M40}&\\
{\tt M41}&Spindle high/low. Accepted, but ignored.\\
{\tt M47}&Repeat program. Since I'm not sure exactly what this should
do, it generates an error.\\
{\tt M48}&\\
{\tt M49}&Feed \& speed override on/off. Accepted, but ignored.\\
{\tt M98}&Call subprogram.\\
{\tt M99}&Return from subprogram. Both of these work.\\
\end{longtable}

The system also accepts line numbers (with {\tt N}), but they don't
have any meaning since none of the valid codes can refer to line numbers.

\section{Shortcomings}

The most obvious shortcoming is that it can be slow to bring up the
image. Once the image is up, it's reasonably fast to move around in
3D. The larger the tool is, the slower it is. If the tool is only
$0.020$ in diameter, then it's plenty fast, but a tool that's
$0.250$ in diameter takes roughly 150 times as long as a tool that's
only $0.020$ in diameter. Having to wait 10 seconds, or even a
minute for the image to come up is possible. In round numbers, with a
quarter inch endmill, you can expect to wait $0.25$ seconds for every
inch of tool travel (at least that's true on my machine). I have ideas
about how to fix this, but it will take some time to make the
necessary changes. As a quick fix, you can change the scale value from
1,000 to something like 100.

The other big problem is the fact that the program is a memory
hog. There are things I can do to improve matters, but some
amount of hoggishness is unavoidable. The program isn't very nice
about its memory needs; it may crash if it can't get the memory it
wants.

You'll notice that, in 3D mode, the image you get isn't really a solid
object at all; it's just the ``skin'' of the object's top. Making the object
appear as a solid is easy, but I find that, given the less than ideal
method of shading that I use now, it's easier to visualize what's been
cut if the object isn't solid.

Objects that are large and/or intricate will be drawn slowly in 3D mode. You'll
notice this for anything larger than about three inches square. I
mentioned above that you can reduce the scale value as a crude
fix for this problem. Doing that makes the internal representation of the
object less detailed. I have ideas about how I can improve the speed, but
implementing these idea will take some time.

The way the object is shaded isn't the greatest. Right now, the
surface of the object is shaded based on the depth of the cut. This
is OK most of the time, but sometimes it would be nice if there were
more contrast between adjacent surfaces. I know how to fix this (I
think), but it's non-trivial.

If you zoom in too far on objects, the program may crash. I know
exactly why this happens, but I'm probably going to totally redo that
part of the program anyway, so I don't want to deal with the problem.

Even for very simple objects, there's an internal limit on the
permitted size of the object. This limit is roughly a cube 32 inches
on each side. I could allow objects that are literally miles to the side
by doubling the memory requirements, but, unfortunately, there's
no middle ground. In any case, my current less efficient methods of
memory usage dictate a limit of roughly a square six inches on a
side.

\section{Bonus}

The ``Test'' menu is there for my own debugging purposes. It's not
meant to be very clean, but there are a few things there that you
might think are interesting.

The output of ``Test Voxel Frame'' looks something like what you see
in the 3D window under Mach3. Every tool position gets a dot in
three-dimensional space, and this draws those dots. The drawing is
cruddy, but it wasn't really meant for public consumption. You can
rotate and zoom using the keyboard. 

Choosing ``Test Pulses'' will show the series of pulses (zeros and
ones) generated by the G-code. These are the signals that would be sent to the
motors of a real-world CNC machine. Again, this was for debugging, and
I think that it only shows the first 100 pulses.

The other choices under the ``Test'' menu have to do with the way that
my program converts G-code to pulses. It's a multi-step process that
begins with the lexer (in computer science lingo, the ``lexical
analyzer''). The lexer converts the string of 
characters that make up the G-code into bite-sized pieces that later
stages find easier to digest. For instance, the lexer converts 
\vskip 0.10cm
\noindent {\tt X0.100 Y1.500} 
\vskip 0.10cm
\noindent into two ``tokens'', one for the X-value and another for the
Y-value. That's probably not so interesting to you, but the other
layers of the process might be.

The parser takes the tokens from the lexer and converts them to
meaningful statements. Basically it verifies that the G-code isn't
absolute gibberish. The output of ``Parsify'' should look pretty much
like the input G-code. 

The next five steps, ``Simplify 00'' through ``Simplify 05'' take the
original G-code and convert it to G-code of a simpler and simpler
form. The 00 step eliminates subroutine calls. The output is exactly what
you would have to type (over and over again) if there were no such
thing as {\tt M98/99}. The 01 step converts everything to the standard
unit of the machine (inches or millimeters) so that {\tt G20/21} no
longer appear in the G-code. The 02 step removes any work offsets so
that {\tt G52} and {\tt G54} through {\tt G59} are eliminated, along
with {\tt G43,44,49} for tool length offset. All coordinate values
generated by step {\tt 01} are adjusted as needed. The 03 step gets
rid of polar coordinates. The 04 step changes all coordinate values to
be absolute so that {\tt G90/91} is not needed in the output, and step
05 adjusts everything for cutter comp.

The output of step 05 is exactly the G-code that you would have to
type to create the same object as the original G-code, but on a
machine that understands only {\tt G00, G01, G02 and G03}, along with
the {\tt M}-codes for tool change and spindle \& coolant control. The ability
to see the output of this final step may be useful from a teaching
point of view. It would help students to visualize what cutter comp and
subroutine calls (for example) really do.





\part{C++/Qt Version}


% Needed to include graphics objects (from metapost).
%\usepackage{graphics}

%\usepackage{epic,eepic} \setlength\maxovaldiam{60pt}

% For ``long'' tables.
%\usepackage{longtable}

%\begin{document}

%\raggedbottom

\newcommand{\ignoretext}[1]{}


\section{What is This Thing?}

For the time being, I am calling this program ``Ger,'' which I got by
combining ``G,'' as in ``G-code,'' with the ``er'' sound of
``verify.'' So, the program is a G-code verifier or a ``Gerifier'' --
``Ger'' for short. You can pronounce as in the name Gary (or Jerry)
when you feel friendly toward the program, or a bear's growl when the 
program isn't doing what you want it to, which shouldn't be very often.

I thought of calling the program ``Gim,'' as in ``G-code simulator,''
but I'll bet that name is taken for something. If you think of a better
name, let me know.

{\bf As you use this program, please let me know if you find anything
  about it that doesn't seem right. Obviously, if it translates your
  code in a way that doesn't make sense, I want to know about that. In
  fact, I'd like to hear about anything odd or unexpected that the
  program does. If you see an inefficiency, inelegance, or just have a
good idea related to the program, then I want to hear about that
too. If you do find any bugs, it will help me if you can
send me the input G-code and/or explain (in detail!) what you did that
led to the strange behavior. I can't fix a bug that I can't
reproduce. I am at {\tt rsfairman@yahoo.com}. }

At this point the program can do two things: translating G-code to a
simpler form and rendering a two-dimensional image of the part being
cut.

\subsection{Translation}

You can give Ger some G-code, and have it spit out a simplified
version of the code you gave it. Here's what I mean.

Fundamentally, any G-code program boils down to having the tool do one
of two things: move linearly or move along a circular arc -- there are
also helical moves, but these are just a combination of a linear move
and an arc move. For example, if the program you input involves cutter
compensation ({\tt G42}), work offsets ({\tt G56}), polar coordinates
({\tt G16}), incremental mode ({\tt G91}), and so forth, then the output
program will have the same effect as the input program -- will move
the tool in the same way and cut the same part -- but it won't involve
any of these more complicated commands. The output program will
involve only {\tt G00, G01, G02} and {\tt G03}.

In fact, there are a few codes in addition to these four that can't
be eliminated from the output program. These additional codes are {\tt
  G17/18/19}, which specify which of the XY, ZX or YZ-plane to work
in, along with the M-codes for spindle start/stop ({\tt M03/04/05}) and
tool change ({\tt M06}). In a section below, there is a list of the
G/M-codes that Ger accepts as input, along with a brief description of
what each one does. 

The translator is useful for several things. It's a good
teaching tool. The more esoteric G-codes can be difficult to
understand, and once you do understand them, it's easy to forget how they
work. By running your G-code through Ger, and looking at the output,
you can check your understanding of particular codes. 

Second, not all controllers are very friendly about certain
G-codes. For instance, I have had bad luck using cutter comp on Mach3.
Maybe it's me, but I've had Mach3 go haywire too many times
when I was using cutter comp to trust it. Using Ger, you can write
your program with cutter comp, have Ger translate it to a program
which does the same thing, but does it without cutter comp, and give
that simpler program to the real-life controller.

Third, even experts get confused. If you're writing a long program,
particularly one that uses sub-programs and incremental mode, it's
easy to get mixed up about where the tool is at a particular point in
the program. Ger expresses its output strictly in absolute mode,
making it easy to check that the tool is where you think it is as the
program proceeds. If the tool is not where you think it should be, then Ger
makes it easy to find out where your program went wrong.

\subsection{2D Rendering}

Ger is able to show a flat (2D) image of the object cut by your
G-code (I'm working on 3D rendering, but it's not ready
yet). The depth of cut is indicated by the shade of gray used, with
darker areas being cut more deeply. 

Once the image is visible, you can drag it around with the mouse or
zoom in and out by using the mouse wheel. Double-click on the
image to bring it back to the standard view showing the entire object
centered in the window. You can use the keyboard instead of the mouse:
use the arrow keys to pan the image, the plus and minus keys to 
zoom in and out, and the enter key to return to the standard view.

\section{Overview of Menus}

When the program is launched a window comes up with several menu
choices. The ``File'' and ``Edit'' menus act as you would expect. You
can have multiple G-code programs open at the same time; each one
appears in its own tab. You can edit and save programs from here too,
although the editor is nothing fancy. You can adjust how much space
each of the left and right panes has by dragging the divider between them.

Use the ``Action'' menu to perform a translation/simplification of
your G-code or to show an image of the object. After you've loaded
some G-code, either by opening a file or by typing it in, choose
``Translate'' and a simplified version of your program will appear in
the right panel. The format is basically the same as ordinary G-code,
although it's been restated to use only 
the most fundamental linear and arc moves. The line numbers that
appear on the right (the N-values) refer to the line in the input file
that lead to that line of output. You can use this to understand where
your input program went wrong -- why didn't the simulated controller
move the tool in the way I thought my G-code specified?

The ``Action'' menu is where you can bring up a various dialog boxes
that set up the simulated machine. The ``Set Geometry'' dialog allows
you to define the dimensions of the raw material being machined,
whether to work in inches or millimeters, the location of the PRZ
(\emph{i.e.}, the origin) and the tool turret. 

The ``Tool Table'' dialog allows you to specify up to 20 tools in the
turret. Ger is able to cut with endmills, ballmills (cutters with a
hemispherical tip), center drills and ordinary drills. In fact, a
``drill'' is treated as if it were an ordinary pointy-ended mill; you
can make linear cuts or plunging cuts with one. Programs change tools
in the ordinary way, by using {\tt M06}. For instance, {\tt M06 T3}
changes to the third tool in the turret. If a program doesn't specify
a particular tool, then it defaults to the first tool in the turret,
which is a quarter-inch endmill.

If your program invokes {\tt G55/56/57/58/59}, then use the ``Work
Offsets'' dialog to set the corresponding values.

Finally, the ``Scale'' dialog allows you to adjust the level of detail
at which the program works. For most uses, it's not necessary for
rendered images to be calculated to 0.001~in -- at that level of
detail a square inch of material would fill an unusually large
computer monitor. If you do need more detail, the ``Scale'' dialog
allows you to get it, although the rendering process may take up to
100 times longer and use 100 times the memory.

To show an image of the part cut by your G-code, choose ``Render --
2D.'' If you modify the G-code, and you want to see the result,
choose ``Render -- 2D'' again.

\section{G-codes and M-codes}

Different controllers use slightly different variations of the
G/M-codes. Ger tries to take a middle-of-the-road approach, although
it is more flexible than many real-world controllers about the order
and grouping of codes. For instance, something like this is accepted:
\begin{verbatim}
G01 X1.0 G21 G01 Y250.0
\end{verbatim}
and would be interpreted as
\begin{verbatim}
G01 X1.0 
G21 
G01 Y250.0
\end{verbatim}
Statements may be terminated by either a new-line (\emph{i.e.}, a
``carriage return'' or ``enter''), or by a semi-colon. Thus, {\tt G01
  X1.0; Y2.0} is valid and 
means something entirely different than {\tt G01 X1.0 Y2.0}.  If it is
possible for \emph{you} to make sense of your program by 
reading the codes one ``chunk'' at a time, then the simulator can
probably make sense of it too, although the controller on the actual
milling machine may be more strict. 

Many of the codes that make sense on a real-world machine are
ignored by the simulator. For instance, {\tt M08, M09} (coolant
on/off) is ignored, and {\tt M47} (repeat program) doesn't really make
sense in this context.

To be clear on how this program interprets the different codes, here's
a list of how they are seen by the program's virtual controller. The
table below is organized by ``actionable chunks,'' meaning a series of inputs
that makes sense together. To make the descriptions more concise, {\tt
  [...]} is used for an optional term, \emph{n} means a number,
possibly with a decimal, and \emph{w} means a whole number.

\vskip 0.25cm
\begin{longtable}{lp{10cm}}
\em{Move} & \tt{[X}\em{n}\tt{]} \tt{[Y}\em{n}\tt{]}
\tt{[Z}\em{n}\tt{]} \tt{[F}\em{n}\tt{]} \\ 
& Notice the use of {\tt [} and {\tt ]}: each of these terms is
optional. \\
& This command moves the tool along a line segment to the given X, Y,
Z, at the given feed rate. If you are in rapid mode ({\tt G00}), then
the {\tt F} is not allowed.\\
{\tt G00}&Rapid travel mode. The program begins in rapid travel
mode. If the very first line is {\tt X1.000 Y1.000}, then the tool
will move to that position in rapid mode.\\
{\tt G01}&Ordinary (not rapid) travel mode.\\
{\tt G02}& Clockwise arc.\\
{\tt G03}&Counter-clockwise arc. \\
& {\tt G02/G03} {\tt [X}\em{n}\tt{]} \tt{[Y}\em{n}\tt{]}
\tt{[Z}\em{n}\tt{]} {\tt [I}\em{n}\tt{]} \tt{[J}\em{n}\tt{]}
\tt{[K}\em{n}\tt{]} \tt{[F}\em{n}{\tt ]} \emph{or} \\
&{\tt G02/G03} {\tt [X}\em{n}\tt{]} \tt{[Y}\em{n}\tt{]} \tt{[Z}\em{n}\tt{]}
\tt{[R}\em{n}\tt{]} \tt{[F}\em{n}\tt{]}\\
&Circular interpolation. These do what you'd expect. Something to
remember: if an arc is specified using {\tt R} rather than {\tt
  I/J/K}, then a positive {\tt R}-value produces an arc that subtends
less than $180^\circ$; using a negative {\tt R}-value produces an arc
that's more than $180^\circ$.\\
&Also, various machines may treat the ZX plane slightly
differently, depending on whether they are ``standard'' or ``vertical
machining.'' So, {\tt G02} and {\tt G03} may be treated in opposite
fashion. Basically, what does ``clockwise'' mean? See Smid, \emph{CNC
  Programming Handbook}, p. 280.\\
&To cut a complete circle, the X/Y/Z values given with the command
must be equal to the starting point of the tool, and you must use
the I/J/K format, not an R-value.\\
&Geometrically, the distance of the center to each of the two
end-points must be the same (that distance is the radius). When using
the I/J/K format, it's not unusual for the center of an arc to require
many digits to specify exactly, perhaps even an infinite number. So
that the programmer only needs to input a reasonable number of digits,
a certain amount of leeway must be allowed. Ger will accept a center
if the distances to the two end-points are within 1\% of each other,
and it will assume that the radius of the circle is equal to the
larger of these two distances.\\
{\tt G04}&Dwell. This is accepted, but everything from {\tt G04} to
the end of the line (or a semi-colon) is ignored.\\
{\tt G15}&Polar coordinates off.\\
{\tt G16}&Polar coordinates on. \\
&You may enter polar coordinate mode while in incremental mode
({\tt G91}) or you may enter incremental mode after invoking {\tt
  G16}, or go back to absolute mode ({\tt G90}). Not every real-world CNC
controller allows this. Ger also allows circular interpolation ({\tt
  G02/G03}) while in polar coordinate mode, another thing that most
real-world machines do not permit. Use of {\tt G01} or {\tt G00} while
in polar coordinate mode is \emph{not} allowed.\\ 
{\tt G17}& Work in the XY-plane (the default).\\
{\tt G18}& Work in the ZX-plane.\\
{\tt G19}& Work in the YZ-plane.\\
& While in cutter comp mode, only the standard XY-plane may be
used. This is typical for real-world controllers.\\
{\tt G20}&Inches.\\
{\tt G21}&Millimeters.\\
& Ger works internally using either inches of millimeters, but you can
freely change from one unit of measure to the other within your
G-code. In fact, Ger is more flexible than many real-world
controllers; you can sprinkle {\tt G20}'s and {\tt G21}'s all over the
place.\\  
{\tt G28}&Machine Zero Return. This is accepted, but everyting from
{\tt G28} to the end of the line (or a semi-colon) is ignored. On the
simulated machine, it has the effect of moving the tool to the
position of the tool turret.\footnote{In fact, I'm not sure whether
  I've implemented this.}\\
{\tt G40}&Cancel cutter comp, also known as ``tool radius compensation.''\\
{\tt G41}&Cutter comp, left.\\
{\tt G42}&Cutter comp, right. \\
&{\tt G41/G42 D}\emph{w} \emph{or}\\
&{\tt G41/G42 H}\emph{w} \\
& The cutter comp is given relative to a value stored in a
D-register or an H-register. The D-register typically contains the
diameter of the tool and the H-register contains a value for tool
wear. These are provided by using the ``Tool Turret'' dialog. As a
reminder, the left/right distinction here refers to the 
position of the tool relative to its path. Thus, {\tt G41} puts the
tool to the left of the path it follows.\\
&Ger does things slightly differently than how many real-world
controllers seem to work. As soon as a {\tt G41/42} appears, the
virtual machine peeks ahead to see what the next statement will be and
it bumps the tool out by the tool radius to be ready to cut the very
first curve with the correct cutter comp. I think this is better than the way
it's typically done, but it could confuse people who are trying to
test against a particular real-world machine. There seems to
be so much variation from one controller to another in how this aspect of
cutter comp is handled, that some degree of mis-match between Ger and
the real world is unavoidable. In any case, the usual advice applies
to \emph{all} machines: invoke cutter comp away from the part and make
at least one move before contacting the part. \\
& Note that Ger does not allow things like\\
& {\tt G41 X\#\# Y\#\# Z\#\# D\#\#}\\
& You can't have any extra stuff between the {\tt G41} and the {\tt D}
or {\tt H}-register specification. Many real-world CNC machines do
allow this.\\
& One last point about cutter comp. Ger allows cutter comp to be used
on interior angles, while many real-world controllers do not. However,
Ger is not as smart as some real-world controllers about how it deals
with this case. For instance, if the tool path is specified to form a
very acute vertex, say $5^\circ$, and the tool makes many small moves (less
than the tool radius) near this vertex, then the tool path in cutter
comp mode might not be what you expect. On the other hand, it's hard
to imagine what a person might intend in a case like this, so it seems
better to let the tool do oddball things as a way of informing him
that he \emph{input} something oddball. \\
& About the only sitation I
can imagine where this would arise is if you feed Ger code that was
posted by a (rather stupid) CAM program. The way to ``fix'' this is to
have infinite look-ahead in cutter comp mode so that Ger can remove
all of those small moves near the apex from the statement stream, but
it's not clear to me that this is something that \emph{should} be
fixed. It seems better for the controller to behave in a way that
makes it apparent that the input G-code is weird, and probably wrong, rather
than acting in a way that hides from the user potential problems with his
G-code. \\
{\tt G43}&Positive tool length offset (TLO).\\
{\tt G44}&Negative tool length offset.\\
{\tt G49}&Cancel tool length offset.\\
&{\tt G43/G44 [Z}\emph{n}{\tt ] H}\emph{w}\\
& The simulator considers the {\tt Z}-value to be
optional, although it is required by many real-world machines. \\
& As a reminder, in the case of {\tt G43}, this has the effect
of adding the value in the H-register to all Z-moves until a {\tt G49}
is reached. If the H-value is negative, then the tool will cut more
deeply. {\tt G44} has the opposite effect.\\
&A program may not enter TLO while in incremental mode, when using
polar coordinates, or while in cutter comp mode. However, once TLO is
in effect, a program may enter any of these modes. If your program
does enter one of these modes, then it must leave these modes before
cancelling TLO. \\
{\tt G52}&Local coordinate system.\\
& {\tt G52 [X}\em{n}\tt{]} \tt{[Y}\em{n}\tt{]} \tt{[Z}\em{n}\tt{]} \\
& This changes the origin to be the point at the given X, Y, Z, stated
relative to the PRZ. All moves after {\tt G52} are as though the PRZ
is at this new position. Each of the X, Y, and Z values is optional;
omitted values are assumed to be zero (no change). This is cancelled
with {\tt G54}. \\
{\tt G54}& Return to normal work offset. That is, use the PRZ as the
origin. If a tool length offset is in force ({\tt G43} or {\tt G44}),
then it remains in force after calling {\tt G54}.\\
{\tt G55}&\\
{\tt G56}&\\
{\tt G57}&\\
{\tt G58}&\\
{\tt G59}&Codes {\tt G54} through {\tt G59} are similar to {\tt G52},
but the values used for X, Y and Z come from the ``work offsets''
table. So, these codes appear alone, simply as {\tt G55} (say),
without any additional codes.\\
&Just as with TLO mode, none of these codes, including {\tt G52}, may be
entered or exited while in incremental mode, polar coordinate mode, or
cutter comp mode.\\
&Normally, the values used for work offsets are the distance from
machine zero to program zero (the PRZ), but the virtual machine has no
``machine zero.'' For the purpose of work offset settings, think of
the PRZ as the machine zero. Put another way, the work offset values
should be the amount in each direction, X, Y and Z, that you want the
tool to be offset from the position it would have without the change
due to applying one of {\tt G55} through {\tt G59}.\\
{\tt G90}&Absolute mode.\\
{\tt G91}&Incremental mode.
\end{longtable}
\vskip 0.50cm
As you can see, none of the canned cycles, like {\tt G81} for
drilling, have been implemented. Some of them might manage to get
through my system without causing an error, but none of them do
anything. I haven't thought very hard about this, but I don't think that
they would be hard to add.

Here are the valid M-codes.
\vskip 0.25cm
\begin{longtable}{lp{10cm}}
{\tt M00}&\\
{\tt M01}&\\
{\tt M02}&These are different forms of ``program stop.'' The simulator
simply halts when it reaches one of these.\\
{\tt M03}&Spindle on, clockwise.\\
&         {\tt M03 S}\emph{w}\\
& where \emph{w} is the spindle speed.\\
{\tt M04}&Spindle on, counter-clockwise. Similar to {\tt M03}. The
program notes internally that the spindle is on, but it has no real
effect.\\
{\tt M05}&Spindle off.\\
{\tt M06}&Tool change.\\
& {\tt M06 T}\emph{w}\\
& changes to the tool in the \emph{w} position of the turret. The
simulator moves the tool in a straight line from its curent position
to the tool turret, changes to the new tool, and stops at that
position.\footnote{These motions don't appear in the translator part of the
  program; they are used to detect tool crashes.} \\
{\tt M07}&\\
{\tt M08}&\\
{\tt M09}&These coolant on/off commands are accepted, but they do
nothing.\\
{\tt M30}&Halts the program.\\
{\tt M40}&\\
{\tt M41}&Spindle high/low. Accepted, but ignored.\\
{\tt M47}&Repeat program. The simulator halts immediately if it
reaches this code.\\
{\tt M48}&\\
{\tt M49}&Feed \& speed override on/off. Accepted, but ignored.\\
{\tt M98}&Call subprogram.\\
&{\tt M98 P}\emph{w} {\tt L\emph{w}}\\
& This calls a sub-program, where {\tt P} gives its number, and {\tt
  L} gives the number of times to call it.\\
{\tt M99}&Return from subprogram.
\end{longtable}

Two other codes are accepted: the {\tt O} code for a program number,
and {\tt N} may be used for line numbers. Any line numbers given with
{\tt N} are ignored. Every program is required to start with an O-value.

If there are any errors in the program, then the line number on
which the errors are detected are given by counting lines in the input
text file, not with reference to any {\tt N}-specified line numbers in
the input G-code. Thus, if an error is reported on line 3 (say), then the
problem was detected on the third line from the top, not the line that
starts with {\tt N3} (assuming there is such a line).

Finally, comments are expressed with parenthesis, {\tt (...)}, as
usual. The simulator accepts multi-line comments, but does not accept
nested comments.





\ignoretext{









\section{Shortcomings}

The most obvious shortcoming is that it can be slow to bring up the
image. Once the image is up, it's reasonably fast to move around in
3D. The larger the tool is, the slower it is. If the tool is only
$0.020$ in diameter, then it's plenty fast, but a tool that's
$0.250$ in diameter takes roughly 150 times as long as a tool that's
only $0.020$ in diameter. Having to wait 10 seconds, or even a
minute for the image to come up is possible. In round numbers, with a
quarter inch endmill, you can expect to wait $0.25$ seconds for every
inch of tool travel (at least that's true on my machine). I have ideas
about how to fix this, but it will take some time to make the
necessary changes. As a quick fix, you can change the scale value from
1,000 to something like 100.

The other big problem is the fact that the program is a memory
hog. There are things I can do to improve matters, but some
amount of hoggishness is unavoidable. The program isn't very nice
about its memory needs; it may crash if it can't get the memory it
wants.

You'll notice that, in 3D mode, the image you get isn't really a solid
object at all; it's just the ``skin'' of the object's top. Making the object
appear as a solid is easy, but I find that, given the less than ideal
method of shading that I use now, it's easier to visualize what's been
cut if the object isn't solid.

Objects that are large and/or intricate will be drawn slowly in 3D mode. You'll
notice this for anything larger than about three inches square. I
mentioned above that you can reduce the scale value as a crude
fix for this problem. Doing that makes the internal representation of the
object less detailed. I have ideas about how I can improve the speed, but
implementing these idea will take some time.

The way the object is shaded isn't the greatest. Right now, the
surface of the object is shaded based on the depth of the cut. This
is OK most of the time, but sometimes it would be nice if there were
more contrast between adjacent surfaces. I know how to fix this (I
think), but it's non-trivial.

If you zoom in too far on objects, the program may crash. I know
exactly why this happens, but I'm probably going to totally redo that
part of the program anyway, so I don't want to deal with the problem.

Even for very simple objects, there's an internal limit on the
permitted size of the object. This limit is roughly a cube 32 inches
on each side. I could allow objects that are literally miles to the side
by doubling the memory requirements, but, unfortunately, there's
no middle ground. In any case, my current less efficient methods of
memory usage dictate a limit of roughly a square six inches on a
side.

\section{Bonus}

The ``Test'' menu is there for my own debugging purposes. It's not
meant to be very clean, but there are a few things there that you
might think are interesting.

The output of ``Test Voxel Frame'' looks something like what you see
in the 3D window under Mach3. Every tool position gets a dot in
three-dimensional space, and this draws those dots. The drawing is
cruddy, but it wasn't really meant for public consumption. You can
rotate and zoom using the keyboard. 

Choosing ``Test Pulses'' will show the series of pulses (zeros and
ones) generated by the G-code. These are the signals that would be sent to the
motors of a real-world CNC machine. Again, this was for debugging, and
I think that it only shows the first 100 pulses.

The other choices under the ``Test'' menu have to do with the way that
my program converts G-code to pulses. It's a multi-step process that
begins with the lexer (in computer science lingo, the ``lexical
analyzer''). The lexer converts the string of 
characters that make up the G-code into bite-sized pieces that later
stages find easier to digest. For instance, the lexer converts 
\vskip 0.10cm
\noindent {\tt X0.100 Y1.500} 
\vskip 0.10cm
\noindent into two ``tokens'', one for the X-value and another for the
Y-value. That's probably not so interesting to you, but the other
layers of the process might be.

The parser takes the tokens from the lexer and converts them to
meaningful statements. Basically it verifies that the G-code isn't
absolute gibberish. The output of ``Parsify'' should look pretty much
like the input G-code. 

The next five steps, ``Simplify 00'' through ``Simplify 05'' take the
original G-code and convert it to G-code of a simpler and simpler
form. The 00 step eliminates subroutine calls. The output is exactly what
you would have to type (over and over again) if there were no such
thing as {\tt M98/99}. The 01 step converts everything to the standard
unit of the machine (inches or millimeters) so that {\tt G20/21} no
longer appear in the G-code. The 02 step removes any work offsets so
that {\tt G52} and {\tt G54} through {\tt G59} are eliminated, along
with {\tt G43,44,49} for tool length offset. All coordinate values
generated by step {\tt 01} are adjusted as needed. The 03 step gets
rid of polar coordinates. The 04 step changes all coordinate values to
be absolute so that {\tt G90/91} is not needed in the output, and step
05 adjusts everything for cutter comp.

The output of step 05 is exactly the G-code that you would have to
type to create the same object as the original G-code, but on a
machine that understands only {\tt G00, G01, G02 and G03}, along with
the {\tt M}-codes for tool change and spindle \& coolant control. The ability
to see the output of this final step may be useful from a teaching
point of view. It would help students to visualize what cutter comp and
subroutine calls (for example) really do.

Actually, I think that keeping the final layer's output in the
production program makes sense. It's a way for a person to try to
figure out why they're getting strange behavior. Things like cutter
comp are confusing.










}















\end{document}





\end{document}
